\chapter{Theoretische Grundlagen}
\label{chap:grundlagen}

Dieses Kapitel legt das theoretische Fundament für die Dimensionierung des Motorcontrollers. Im Fokus stehen das elektromechanische Modell des BLDC-Motors, die sensorbasierte Kommutierung sowie die leistungselektronischen Verlust- und Temperaturmodelle, welche die hardwareseitigen Grenzen des Systems definieren. Die hier hergeleiteten Zusammenhänge bilden die Basis für die Bauteilauswahl in Kapitel \ref{chap:entwurf_leistung}.

% ============================================================
% 2.1 Der BLDC Motor
% ============================================================
\section{Der BLDC-Motor}
\label{sec:bldc_motor}

Der BLDC-Motor stellt eine technologische Weiterentwicklung der konventionellen Gleichstrommaschine dar, bei der die mechanische Kommutierung durch eine elektronische Steuerung ersetzt wird. Im Gegensatz zu bürstenbehafteten Motoren, bei denen Kohlebürsten und ein Kollektor für die Stromwendung sorgen, erfolgt die Ansteuerung der Statorwicklungen hier in Abhängigkeit von der Rotorlage. Dies eliminiert den mechanischen Verschleiß und die Funkenbildung, was zu einer höheren Zuverlässigkeit, geringerem Wartungsaufwand und einer verbesserten thermischen Charakteristik führt, da die Verlustwärme primär im außenliegenden Stator entsteht und effektiv abgeführt werden kann \cite{MPS_BLDC_AN}.

Technisch betrachtet handelt es sich bei einem BLDC-Motor um eine permanentmagneterregte Synchronmaschine (PMSM). Der Begriff DC bezieht sich hierbei nicht auf das interne elektromagnetische Wirkprinzip, sondern auf die externe Speisung aus einer Gleichspannungsquelle, die durch den Inverter in die benötigten Phasenströme umgewandelt wird.

\subsection{Aufbau und physikalisches Wirkprinzip}
Der mechanische Aufbau gliedert sich primär in den feststehenden Stator und den rotierenden Rotor, wie der Querschnitt in Abbildung \ref{fig:bldc_aufbau} schematisch veranschaulicht.

\paragraph{Stator}
Der Stator besteht aus geschichteten Stahlblechen (Lamellen), um Wirbelstromverluste zu minimieren. In den Nuten des Stators sind die Wicklungen untergebracht, die meist in drei Phasen ($U, V, W$) angeordnet sind. Diese Wicklungen sind im Stern verschaltet, wobei der Sternpunkt im Motor isoliert ist und nicht nach außen herausgeführt wird.

\paragraph{Rotor}
Der Rotor trägt die Permanentmagnete. Je nach Anordnung unterscheidet man zwischen Innenläufern, bei denen der Rotor innerhalb des Stators dreht, und Außenläufern. Die Anzahl der magnetischen Pole ($N$ und $S$) variiert je nach Applikation, ist jedoch stets geradzahlig. Eine höhere Polpaarzahl führt bei gleicher elektrischer Frequenz zu einer geringeren mechanischen Drehzahl, erhöht jedoch gleichzeitig das verfügbare Drehmoment.

Das Drehmoment entsteht durch die Lorentzkraft, die auf die bestromten Leiter im Magnetfeld wirkt. Damit eine kontinuierliche Rotation gewährleistet ist, muss das magnetische Feld des Stators dem des Rotors stets um $90^\circ$ vorauseilen.

% \begin{figure}[htbp]
%     \centering
%     \includegraphics[width=0.7\textwidth]{BLDC_1.png}
%     \caption{Querschnitt und prinzipieller Aufbau eines BLDC-Motors \cite{MPS_BLDC_AN}.}
%     \label{fig:bldc_aufbau}
% \end{figure}

\begin{figure}[htbp]
    \centering
    \begin{tikzpicture}
        % Bild
        \node[inner sep=0] (img) {\includegraphics[width=0.7\textwidth]{BLDC_1.png}};
        
        % Entwurf-Banner (halbtransparent, diagonal)
        \node at (img.center) [
            rotate=0,
            scale=5,
            text opacity=0.35,  % macht es dezent
            color=red
        ] {DRAFT};
    \end{tikzpicture}
    \caption{Querschnitt und prinzipieller Aufbau eines BLDC-Motors \cite{MPS_BLDC_AN}.}
    \label{fig:bldc_aufbau}
\end{figure}

\subsection{Elektrisches Ersatzmodell und BEMF}
Für die steuerungstechnische Betrachtung und die Dimensionierung der Endstufe wird jede Phase des Motors durch ein elektrisches Ersatzschaltbild (ESB) modelliert (siehe Abbildung \ref{fig:bldc_phase}). Dieses besteht aus einer Reihenschaltung des ohmschen Wicklungswiderstands $R$, der Wicklungsinduktivität $L$ und einer induzierten Spannungsquelle, der sogenannten Gegen-Elektromotorischen Kraft (Back Electromotive Force, BEMF).

\begin{figure}[htbp]
    \centering
    \begin{tikzpicture}
        % Paths, nodes and wires:
        \node[ocirc](N1) at (1, 4.941){} node[anchor=west] at (N1.east){$+$};
        \node[ocirc](N2) at (6, 4.941){} node[anchor=east] at (N2.west){$-$};
        \draw (1, 7) to[european resistor, /tikz/circuitikz/bipoles/length=0.840cm, l_={$R$}] (3, 7);
        \draw (2.375, 7) to[american inductor, /tikz/circuitikz/bipoles/length=0.840cm, l_={$L$}, label distance=0.14cm] (4.625, 7);
        \draw (4.5, 7) to[european voltage source, /tikz/circuitikz/bipoles/length=0.700cm, l_={$e_U$}] (5.75, 7);
        \draw (5.5, 7) -- (6, 7) -| (6, 5);
        \draw (1, 5) -| (1, 7);
        \node[shape=rectangle, minimum width=0.965cm, minimum height=0.465cm] at (0.75, 6.5){} node[anchor=north west, align=left, text width=0.577cm, inner sep=6pt] at (0.25, 6.75){\textcolor{rgb,255:red,227;green,36;blue,0}{\small $i_U$}};
        \path[draw={rgb,255:red,227;green,36;blue,0}, line width=1pt, -stealth] (1, 6.436) -| (1, 6.5);
        \node[shape=rectangle, minimum width=1cm, minimum height=0.5cm] at (3.75, 5.25){} node[anchor=north west, align=left, text width=0.647cm, inner sep=5pt] at (3.25, 5.5){\textcolor{rgb,255:red,0;green,97;blue,255}{\small $u_U$}};
        \path[draw={rgb,255:red,0;green,97;blue,255}, line width=1pt, -stealth] (1.743, 4.944) -- (5.243, 4.944);
    \end{tikzpicture}
    \caption{Einphasiges ESB des BLDC-Motors.}
    \label{fig:bldc_phase}
\end{figure}

Unter Anwendung des Maschensatzes ergibt sich die Spannungsgleichung für eine einzelne Phase (z.\,B. Phase $U$) gemäß Gleichung \ref{eq:phase_voltage}:

\begin{equation}
    v_U(t) = R \cdot i_U(t) + L \cdot \frac{d i_U(t)}{dt} + e_U(t)
    \label{eq:phase_voltage}
\end{equation}

Hierbei ist $v_U$ die anliegende Klemmenspannung, $i_U$ der Phasenstrom und $e_U$ die induzierte Gegenspannung. 


\paragraph{Bedeutung der Parameter für die Auslegung:}
\begin{itemize}
    \item \textbf{Widerstand $R$:} Bestimmt die statischen Kupferverluste ($P = I^2R$) und damit die Erwärmung des Motors.
    \item \textbf{Induktivität $L$:} Glättet den Stromverlauf bei PWM-Ansteuerung. Eine zu kleine Induktivität führt zu hohem Stromrippel, was die Kondensatoren im Zwischenkreis belastet. Die elektrische Zeitkonstante $\tau_{\mathrm{el}} = L/R$ ist ein Maß für die Stromanstiegsgeschwindigkeit.
    \item \textbf{BEMF $e_U$:} Begrenzt die maximale Drehzahl. Wenn die BEMF die Versorgungsspannung erreicht, kann kein Strom mehr in den Motor getrieben werden.
\end{itemize}

Die BEMF entsteht durch die Rotation des magnetischen Rotorfeldes relativ zu den Statorwicklungen (Induktionsgesetz). Nach \cite{Pillay1991} lässt sich die Amplitude der BEMF direkt proportional zur Winkelgeschwindigkeit $\omega_m$ des Rotors gemäß Gleichung \ref{eq:bemf} beschreiben:

\begin{equation}
    e(t) = k_e \cdot \omega_m \cdot f(\theta_e)
    \label{eq:bemf}
\end{equation}

Dabei ist $k_e$ die spezifische BEMF-Konstante des Motors. Charakteristisch für BLDC-Motoren ist der trapezförmige Verlauf der Funktion $f(\theta_e)$, resultierend aus der rechteckigen Verteilung des magnetischen Flusses im Luftspalt. Dies unterscheidet den BLDC-Motor von der gewöhnlichen PMSM, welche eine sinusförmige BEMF aufweist. 

Um ein konstantes Drehmoment zu erzeugen, sollte der Phasenstrom idealerweise rechteckförmig in die Phasen eingeprägt werden. So korrespondieren die Stromblöcke exakt mit dem flachen Dach der trapezförmigen BEMF, was in Abbildung \ref{fig:bemf_trapez} grafisch verdeutlicht wird.

% \begin{figure}[htbp]
%     \centering
%     \includegraphics[width=0.7\textwidth]{BEMF.png}
%     \caption{Verlauf der BEMF und der korrespondierenden Blockströme \cite{MPS_BLDC_AN}.}
%     \label{fig:bemf_trapez}
% \end{figure}

\begin{figure}[htbp]
    \centering
    \begin{tikzpicture}
        % Bild
        \node[inner sep=0] (img) {\includegraphics[width=0.7\textwidth]{BEMF.png}};
        
        % Entwurf-Banner
        \node at (img.center) [
            rotate=0,
            scale=5,
            text opacity=0.35,
            color=red
        ] {DRAFT};
    \end{tikzpicture}
    \caption{Verlauf der BEMF und der korrespondierenden Blockströme \cite{MPS_BLDC_AN}.}
    \label{fig:bemf_trapez}
\end{figure}

% ============================================================
% 2.2 Kommutierung
% ============================================================
\section{Kommutierung und Rotorlageerfassung}
\label{sec:kommutierung}

Damit der Motor ein kontinuierliches Drehmoment erzeugt, müssen die Statorwicklungen synchron zur aktuellen Rotorposition bestromt werden. Ziel der Kommutierung ist es, den Winkel zwischen dem Erregerfeld des Rotors und dem Statorfeld im Bereich von $90^\circ$ elektrisch zu halten, da hierbei die maximale Lorentzkraft wirkt.

\subsection{Hall-Sensorik}
Im vorliegenden Projekt wird zur Detektion der Rotorlage ein sensorbasiertes Verfahren eingesetzt. Dazu befinden sich im Stator drei Hall-Effekt-Sensoren ($H_1, H_2, H_3$), die das Magnetfeld der vorbeidrehenden Rotorpole erfassen. Üblicherweise sind diese Sensoren räumlich um $120^\circ$ elektrisch versetzt angeordnet. Dieser Versatz führt dazu, dass pro elektrischer Umdrehung ($360^\circ$) sechs diskrete Zustände (Sektoren) eindeutig unterscheidbar sind.

\subsection{Blockkommutierung}
Das gängigste Ansteuerverfahren für BLDC-Motoren ist die Blockkommutierung. Basierend auf den digitalen Signalen der Hall-Sensoren ordnet die Steuerelektronik jedem erfassten Sektor ein festes Bestromungsmuster zu.
Charakteristisch ist hierbei, dass zu jedem Zeitpunkt immer nur zwei der drei Motorphasen aktiv bestromt werden: Eine Phase wird auf das Potential der Zwischenkreisspannung $V_{\mathrm{DC}}$ gezogen, eine auf das Massepotential (GND), während die dritte Phase unbeschaltet (floating) bleibt. Die sich daraus ergebende Kommutierungslogik ist für alle sechs Sektoren in Tabelle \ref{tab:kommutierung} übersichtlich zusammengefasst.

\begin{table}[htbp]
    \centering
    \caption{Zustandstabelle der Blockkommutierungslogik in Abhängigkeit der Hall-Sensoren}
    \label{tab:kommutierung}
    \begin{tabular}{ccc ccc c}
        \toprule
        \multicolumn{3}{c}{Hall-Sensoren} & 
        \multicolumn{3}{c}{Phasenzustand} & 
        Sektor \\
        \cmidrule(lr){1-3} \cmidrule(lr){4-6}
        $H_1$ & $H_2$ & $H_3$ & $U$ & $V$ & $W$ & \\
        \midrule
        1 & 0 & 1 & $\mathrm{V_{DC}}$ & GND & NC & I \\
        1 & 0 & 0 & $\mathrm{V_{DC}}$ & NC & GND & II \\
        1 & 1 & 0 & NC & $\mathrm{V_{DC}}$ & GND & III \\
        0 & 1 & 0 & GND & $\mathrm{V_{DC}}$ & NC & IV \\
        0 & 1 & 1 & GND & NC & $\mathrm{V_{DC}}$ & V \\
        0 & 0 & 1 & NC & GND & $\mathrm{V_{DC}}$ & VI \\
        \bottomrule
    \end{tabular}
\end{table}

% ============================================================
% 2.3 Brückentopologie
% ============================================================
\section{Leistungselektronische Ansteuerung}
\label{sec:topologie}

Die physikalische Umsetzung der in Tabelle \ref{tab:kommutierung} definierten Schaltzustände erfolgt über eine dreiphasige Wechselrichterbrücke, oft auch als B6-Brücke bezeichnet.

\subsection{Brückentopologie}
Das Prinzipschaltbild dieser Leistungsendstufe ist in Abbildung \ref{fig:three_phase_bridge} dargestellt. Die Schaltung besteht aus drei parallel am Gleichspannungszwischenkreis ($V_{\mathrm{DC}}$) liegenden Halbbrücken. Jede dieser Halbbrücken verfügt über zwei Leistungsschalter (MOSFETs):
\begin{itemize}
    \item \textbf{High-Side-MOSFET (HS):} Verbindet den Phasenausgang mit der Versorgungsspannung $V_{\mathrm{DC}}$.
    \item \textbf{Low-Side-MOSFET (LS):} Verbindet den Phasenausgang mit dem Bezugspotential GND.
\end{itemize}
In modernen Niederspannungsanwendungen werden fast ausschließlich N-Kanal-MOSFETs eingesetzt, da diese bei gleicher Chipfläche einen geringeren Einschaltwiderstand $R_{\mathrm{DS(on)}}$ aufweisen als vergleichbare P-Kanal-Typen.

\begin{figure}[htbp]
    \centering
    \begin{tikzpicture}
        % Paths, nodes and wires:
        \node[nigfete, bodydiode] at (3, 7.008){};
        \node[nigfete, bodydiode] at (3, 5){};
        \node[nigfete, bodydiode] at (5, 7.008){};
        \node[nigfete, bodydiode] at (5, 5){};
        \node[nigfete, bodydiode] at (7, 7){};
        \node[nigfete, bodydiode] at (7, 4.992){};

        \draw (3, 7.778) -- (3, 8.25) -- (5, 8.25) -| (5, 7.778);
        \draw (7, 7.75) -| (7, 8.25) -- (5, 8.25);
        \draw (3, 4.25) -| (3, 3.75) -- (5, 3.75) -| (5, 4.23);
        \draw (7, 4.222) -| (7, 3.75) |- (5, 3.75);

        \draw (3, 6.238) -- (3, 5.75);
        \draw (5, 6.238) -| (5, 5.77);
        \draw (7, 6.25) -| (7, 5.75);

        \node[circ](N1) at (3, 6){} node[anchor=west] at ([xshift=0.04cm]N1.east){$U$};
        \node[circ](N2) at (5, 6){} node[anchor=west] at (N2.east){$V$};
        \node[circ](N3) at (7, 6){} node[anchor=west] at (N3.east){$W$};

        \draw (1, 7) to[european voltage source, 
              /tikz/circuitikz/bipoles/length=0.910cm, 
              l={$36\,\mathrm{V}$}] (1, 5);

        \draw (1, 7) -| (1, 8.25) -- (3, 8.25);
        \draw (1, 5) -| (1, 3.75) -- (3, 3.75);
    \end{tikzpicture}
    \caption{Prinzipschaltbild einer B6-Brücke zur Ansteuerung des Motors.}
    \label{fig:three_phase_bridge}
\end{figure}

\subsection{Ansteuerung und Bootstrap-Prinzip}
Die Verwendung von N-Kanal-MOSFETs im HS-Pfad erfordert zum vollständigen Durchsteuern eine Gate-Spannung, die deutlich über der Zwischenkreisspannung liegt. Diese benötigte Überspannung wird üblicherweise durch eine Bootstrap-Schaltung generiert. Dabei lädt sich ein Bootstrap-Kondensator über eine Diode auf die Treiber-Versorgungsspannung $V_{\mathrm{CC}}$ auf, solange der LS-MOSFET leitend ist. Schaltet die Halbbrücke um, „schwimmt“ das Potential des Kondensators mit der Source-Spannung des HS-MOSFETs nach oben und stellt dem Gate-Treiber so die nötige Spannung bereit \cite{AN6076}.

\subsection{Pulsweitenmodulation (PWM)}
Um nicht nur die Richtung, sondern auch die Amplitude des Stroms (und damit das Drehmoment) kontinuierlich zu steuern, werden die Schalter nicht rein statisch eingeschaltet, sondern hochfrequent getaktet. Durch die Variation des Tastverhältnisses $d$ (Duty Cycle) ergibt sich die am Motor anliegende effektive Phasenspannung proportional zur Zwischenkreisspannung als $\bar{v}_{\mathrm{ph}} = d \cdot V_{\mathrm{DC}}$.

% ============================================================
% 2.4 Modulationsverfahren (SVPWM)
% ============================================================
\section{Modulationsverfahren}
\label{sec:modulation}



Während die Blockkommutierung (Kapitel \ref{sec:kommutierung}) lediglich festlegt, welche Phasen prinzipiell bestromt werden, bestimmt das Modulationsverfahren die präzise zeitliche Verteilung der Schaltzustände zur Generierung der gewünschten Ausgangsspannung. Neben der einfachen Sinus-Pulsweitenmodulation (SPWM) hat sich die Raumzeigermodulation (Space Vector PWM, SVPWM) als Industriestandard für dreiphasige Antriebe etabliert.

\subsection{Grundlagen der Raumzeigermodulation}
Im Gegensatz zur SPWM, welche die drei Halbbrücken isoliert betrachtet, modelliert die SVPWM den dreiphasigen Wechselrichter als ein Gesamtsystem, das einen rotierenden Spannungsraumzeiger $\vec{v}_\mathrm{ref}$ in der komplexen Ebene erzeugt \cite{Xia2012,Nahin2022-ts}.

Da jede der drei Halbbrücken exakt zwei Zustände annehmen kann (High oder Low), ergeben sich für die B6-Brücke insgesamt $2^3 = 8$ mögliche diskrete Schaltzustände. Diese lassen sich in zwei Kategorien unterteilen:
\begin{itemize}
    \item \textbf{Sechs aktive Vektoren ($\vec{V}_1$ bis $\vec{V}_6$):} Diese Vektoren spannen in der komplexen Ebene ein regelmäßiges Sechseck (Hexagon) auf und besitzen eine konstante Amplitude von $\frac{2}{3} V_{\mathrm{DC}}$.
    \item \textbf{Zwei Nullvektoren ($\vec{V}_0$ und $\vec{V}_7$):} Bei $\vec{V}_0$ (000) sind alle drei unteren Schalter leitend, bei $\vec{V}_7$ (111) alle drei oberen. In beiden Fällen liegt an den Motorphasen keine Potentialdifferenz an, weshalb die verkettete Ausgangsspannung null beträgt.
\end{itemize}

Die Lage der aktiven Vektoren sowie die Geometrie des resultierenden Hexagons sind in Abbildung \ref{fig:svpwm} grafisch dargestellt.

% \begin{figure}[htbp]
%     \centering
%     \includegraphics[width=0.7\textwidth]{SVPWM.png}
%     \caption{Raumzeigerdiagramm der SVPWM \cite{Nahin2022-ts}.}
%     \label{fig:svpwm}
% \end{figure}


\begin{figure}[htbp]
    \centering
    \begin{tikzpicture}
        % Bild
        \node[inner sep=0] (img) {\includegraphics[width=0.7\textwidth]{SVPWM.png}};
        
        % Entwurf-Banner
        \node at (img.center) [
            rotate=0,
            scale=5,
            text opacity=0.35,
            color=red
        ] {DRAFT};
    \end{tikzpicture}
    \caption{Raumzeigerdiagramm der SVPWM \cite{Nahin2022-ts}.}
    \label{fig:svpwm}
\end{figure}

\subsection{Synthese des Referenzspannungszeigers}
Jeder beliebige Zielspannungszeiger $\vec{v}_{\mathrm{ref}}$, der sich innerhalb des Hexagons befindet, kann durch eine zeitliche Mittelung (Pulsweitenmodulation) der beiden benachbarten aktiven Vektoren sowie der Nullvektoren gebildet werden.
Befindet sich der Referenzzeiger beispielsweise im ersten Sektor zwischen $\vec{V}_1$ und $\vec{V}_2$, so muss die Integration über eine Periodendauer $T_{\mathrm{PWM}}$ gemäß Gleichung \ref{eq:svpwm_integral} erfüllt sein:

\begin{equation}
\int\limits_{0}^{T_{\mathrm{PWM}}} \vec{v}_{\mathrm{ref}} \, dt 
= \int\limits_{0}^{T_1} \vec{V}_1 \, dt 
  + \int\limits_{T_1}^{T_1+T_2} \vec{V}_2 \, dt 
  + \int\limits_{T_1+T_2}^{T_{\mathrm{PWM}}} \vec{V}_0 \, dt
    \label{eq:svpwm_integral}
\end{equation}

Hierbei definieren $T_1$, $T_2$ und $T_0$ die jeweiligen Einschaltzeiten der Vektoren innerhalb einer PWM-Periode. Durch eine geschickte Anordnung der Schaltsequenzen, beispielsweise durch symmetrische Ausrichtung um die Periodenmitte (Center-Aligned PWM), lässt sich der Oberschwingungsgehalt im Ausgangsstrom maßgeblich minimieren.

\subsection{Vorteile gegenüber klassischer PWM}
Der entscheidende elektrotechnische Vorteil der SVPWM gegenüber der sinusförmigen PWM liegt in der besseren Ausnutzung der zur Verfügung stehenden Zwischenkreisspannung.
Während bei der SPWM die maximale Ausgangsspannung durch die Phasenspannung starr limitiert ist, erlaubt die SVPWM durch die Überlagerung einer Nullsystemkomponente (welche im Phasenspannungsverlauf als Sattelform bzw. 3. Harmonische sichtbar wird) eine höhere Amplitude der tatsächlichen, verketteten Motorspannung.

Der maximale lineare Aussteuergrad erhöht sich dabei nach Gleichung \ref{eq:svpwm_gain} um den Faktor:

\begin{equation}
    \frac{V_{\mathrm{SVPWM,max}}}{V_{\mathrm{SPWM,max}}} = \frac{1}{\cos(30^\circ)} = \frac{2}{\sqrt{3}} \approx 1,155
    \label{eq:svpwm_gain}
\end{equation}

Dies entspricht einem Spannungszugewinn von ca. 15,5 \%. Da in der vorliegenden Applikation eine vergleichsweise niedrige Zwischenkreisspannung zur Verfügung steht, ist diese Spannungsreserve essenziell, um die volle Leistungsfähigkeit des Motors bei hohen Drehzahlen abzurufen \cite{Xia2012}.

% ============================================================
% 2.5 Verluste
% ============================================================
\section{Verlustmechanismen in MOSFETs}
\label{sec:verluste}

Ein idealer Halbleiter würde verlustfrei arbeiten. Reale Leistungstransistoren weisen jedoch parasitäre Eigenschaften auf, die unvermeidlich zu einer Verlustleistung führen. Diese elektrische Leistung wird in Wärmeenergie umgewandelt und muss abgeführt werden, um die thermische Zerstörung des Bauteils zu verhindern. Die Gesamtverluste $P_{\mathrm{tot}}$ eines MOSFETs setzen sich primär aus Leit- und Schaltverlusten zusammen \cite{Infineon_MOSFET_Losses}.

\subsection{Leitverluste}
Im statisch eingeschalteten Zustand verhält sich der Drain-Source-Kanal des MOSFETs näherungsweise wie ein ohmscher Widerstand $R_{\mathrm{DS(on)}}$. Die daraus resultierenden Leitverluste lassen sich in Abhängigkeit des Effektivstroms gemäß Gleichung \ref{eq:p_cond} berechnen:

\begin{equation}
    P_{\mathrm{cond}} = I_{\mathrm{rms}}^2 \cdot R_{\mathrm{DS(on)}}(T_\mathrm{J})
    \label{eq:p_cond}
\end{equation}

Kritisch für die Systemstabilität ist hierbei die Temperaturabhängigkeit: Der Widerstandswert $R_{\mathrm{DS(on)}}$ steigt mit zunehmender Sperrschichttemperatur $T_\mathrm{J}$ signifikant an. Dies führt zu einem positiven thermischen Rückkopplungseffekt, der bei der Dimensionierung des Kühlkonzepts zwingend berücksichtigt werden muss.

\subsection{Schaltverluste}
Da MOSFETs bauartbedingt nicht unendlich schnell umschalten können, treten beim Übergang zwischen dem leitenden und sperrenden Zustand Transienten auf. In diesen kurzen Phasen liegen am Halbleiter zeitgleich hohe Spannungen und Ströme an.
Unter Annahme einer induktiven Last lassen sich die Summe aus Einschalt- ($P_{\mathrm{on}}$) und Ausschaltverlusten ($P_{\mathrm{off}}$) näherungsweise als lineare Funktion der Schaltfrequenz $f_{\mathrm{SW}}$ nach Gleichung \ref{eq:p_sw} beschreiben:

\begin{equation}
    P_{\mathrm{sw}} \approx \frac{1}{2} V_{\mathrm{DC}} \cdot I_{\mathrm{D}} \cdot (t_\mathrm{r} + t_\mathrm{f}) \cdot f_{\mathrm{SW}}
    \label{eq:p_sw}
\end{equation}

Hierbei repräsentieren $t_\mathrm{r}$ (Rise Time) und $t_\mathrm{f}$ (Fall Time) die realen Schaltzeiten, welche maßgeblich von der Treiberleistung und den gewählten Gate-Vorwiderständen definiert werden. Es wird ersichtlich, dass eine Erhöhung der PWM-Frequenz zur Reduktion von Stromrippeln unweigerlich zu proportional steigenden Schaltverlusten führt.

% ============================================================
% 2.6 Thermik
% ============================================================
\section{Thermisches Modell}
\label{sec:thermik}

Um die Betriebssicherheit der Endstufe zu gewährleisten, muss sichergestellt werden, dass die maximal zulässige Sperrschichttemperatur $T_{\mathrm{J,max}}$ der Halbleiter unter keinen Umständen überschritten wird. Hierzu bedient man sich eines thermischen Modells. Dabei wird angenommen, dass der entstandene Wärmestrom ($P_{\mathrm{tot}}$) von der innersten Wärmequelle (Junction) über verschiedene thermische Widerstände ($R_{\mathrm{th}}$) bis zur Umgebungsluft (Ambient) abfließt.

Im eingeschwungenen, stationären Zustand lässt sich die resultierende Sperrschichttemperatur nach Gleichung \ref{eq:thermik} berechnen:

\begin{equation}
    T_\mathrm{J} = T_\mathrm{A} + P_{\mathrm{tot}} \cdot (R_{\mathrm{th,JC}} + R_{\mathrm{th,CS}} + R_{\mathrm{th,SA}})
    \label{eq:thermik}
\end{equation}

Die partiellen Wärmewiderstände beschreiben dabei folgende physikalische Übergänge:
\begin{itemize}
    \item $R_{\mathrm{th,JC}}$ (Junction-to-Case): Der innere Wärmewiderstand des MOSFET-Gehäuses, spezifiziert im Datenblatt.
    \item $R_{\mathrm{th,CS}}$ (Case-to-Sink): Der thermische Übergangswiderstand des Montagematerials, beispielsweise einer Wärmeleitfolie oder Wärmeleitpaste.
    \item $R_{\mathrm{th,SA}}$ (Sink-to-Ambient): Der thermische Widerstand des Kühlkörpers zur umgebenden Luft, abhängig von Oberfläche und Luftstrom.
\end{itemize}

Dieses thermische System kann analog zu einem elektrischen Netzwerk in einer thermischen Ersatzschaltung abgebildet werden, wie Abbildung \ref{fig:thermal_esb} verdeutlicht. Es ermöglicht die einfache Berechnung der Temperaturabfälle mittels der Kirchhoffschen Maschengleichungen.

\begin{figure}[htbp]
    \centering
    \begin{tikzpicture}
        % Paths, nodes and wires:
        \draw (6, -6) to[european current source, l={$P_\mathrm{tot}$}] (6, -9);
        \draw (6, -6) to[european resistor, l={$R_\mathrm{th,JC}$}] (9, -6);
        \draw (9, -6) to[european resistor, l={$R_\mathrm{th,CS}$}] (12, -6);
        \draw (12, -9) to[european resistor, l={$R_\mathrm{th,SA}$}] (12, -6);
        \draw (6, -9) -- (12, -9);
        \node[ground] at (12, -9){};
        \node[circ](N1) at (6, -6){} node[anchor=south] at (N1.north){$T_\mathrm{J}$};
        \node[circ](N2) at (12, -9){} node[anchor=west] at (N2.east){$T_\mathrm{A}$};
    \end{tikzpicture}
    \caption{Stationäres thermisches ESB eines gekühlten MOSFETs.}
    \label{fig:thermal_esb}
\end{figure}


Die Aufstellung dieser Wärmekette verdeutlicht, dass eine rein elektrische Betrachtung für die Auslegung von Leistungselektronik nicht ausreichend ist. Erst die Verknüpfung des elektrischen Verlustmodells mit der thermischen Analyse validiert die ausgewählten Komponenten für den Nennbetrieb.