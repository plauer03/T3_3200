\chapter{Theoretische Grundlagen der Ansteuerung}
\label{chap:grundlagen}
\todo[inline, color=green!40]{Zielumfang: ca. 4-5 Seiten. Hier mathematisches Verständnis beweisen!}

\section{Der bürstenlose Gleichstrommotor (BLDC)}
\todo[inline]{Kurz: Statoraufbau, Rotorlageerkennung via Hall-Sensoren. Ersatzschaltbild (R, L, BEMF).}

\section{Topologie des B6-Wechselrichters}
Die Ansteuerung erfolgt über eine 3-phasige Brückenschaltung (B6-Topologie).
\todo[inline]{Erklären: Kommutierungszelle. Was passiert in der Totzeit? (Freilaufdioden-Strom).}
% Hier Bild einfügen: 

\section{Raumzeigermodulation (SVPWM)}
Im Gegensatz zur blockförmigen Kommutierung ermöglicht die SVPWM eine bessere Ausnutzung der Zwischenkreisspannung um den Faktor $2/\sqrt{3} \approx 1,15$.
\todo[inline]{Das Hexagon-Diagramm einfügen. Sektoren erklären.}

Die Einschaltzeiten $T_1, T_2$ und $T_0$ (Nullzeiger) für einen Referenzvektor $V_{ref}$ im Sektor 1 berechnen sich trigonometrisch:
\begin{equation}
    T_1 = \frac{\sqrt{3} \cdot T_s \cdot |V_{ref}|}{V_{DC}} \cdot \sin\left(\frac{\pi}{3} - \theta\right)
\end{equation}
\begin{equation}
    T_2 = \frac{\sqrt{3} \cdot T_s \cdot |V_{ref}|}{V_{DC}} \cdot \sin(\theta)
\end{equation}
\begin{equation}
    T_0 = T_s - T_1 - T_2
\end{equation}