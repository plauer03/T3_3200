\chapter{Hardware-Entwurf und Dimensionierung}
\label{chap:hardware}

Ziel des Hardware-Entwurfs ist die Realisierung einer robusten 3-Phasen-Brückenschaltung (B6-Topologie), die für einen Nennstrom von \SI{30}{\ampere} bei einer Zwischenkreisspannung von \SI{36}{\volt} ausgelegt ist. Im folgenden Kapitel werden die zentralen Komponenten die Leistungshalbleiter und deren Ansteuerung ausgewählt und dimensioniert.

\section{Dimensionierung der Leistungsendstufe}
Die Auswahl der Schaltelemente ist entscheidend für den Wirkungsgrad und die thermische Stabilität des Controllers. Aufgrund der Anforderungen an hohe Schaltgeschwindigkeiten und geringe Verluste kommen N-Kanal-MOSFETs zum Einsatz.

\subsection{Auswahl des MOSFETs}
Für diese Anwendung fiel die Wahl auf den \textbf{Infineon IPP034N08N5} im TO-220-Gehäuse. Diese Entscheidung basiert auf einer Analyse der kritischen Parameter Spannungsfestigkeit und Einschaltwiderstand.

Da die Ladeschlussspannung des \SI{36}{\volt}-Akkus bis zu \SI{42}{\volt} betragen kann und beim generatorischen Bremsen oder durch parasitäre Induktivitäten Spannungsspitzen auftreten, ist eine Sperrspannung ($V_{DS}$) deutlich oberhalb der Batteriespannung notwendig. Der IPP034N08N5 bietet mit \SI{80}{\volt} eine Sicherheitsreserve von fast Faktor 2, was den Verzicht auf komplexe Netzwerke ermöglicht.

Ein weiteres entscheidendes Kriterium ist der statische Drain-Source-Einschaltwiderstand ($R_{DS(on)}$). Mit einem Maximalwert von \SI{3,4}{\milli\ohm} gehört dieser Transistor zu den effizientesten seiner Klasse. Ein geringer $R_{DS(on)}$ ist essenziell, da die Durchlassverluste quadratisch mit dem Laststrom steigen.

Die wesentlichen Kennwerte sind in Tabelle \ref{tab:mosfet_data} zusammengefasst.

\begin{table}[h!]
    \centering
    \caption{Zentrale Parameter des IPP034N08N5}
    \label{tab:mosfet_data}
    \begin{tabular}{l l r}
        \toprule
        Parameter & Symbol & Wert \\
        \midrule
        Max. Drain-Source-Spannung & $V_{DS}$ & \SI{80}{\volt} \\
        Max. Dauerstrom ($T_C = 25^\circ C$) & $I_D$ & \SI{120}{\ampere} \\
        Einschaltwiderstand & $R_{DS(on)}$ & \SI{3,4}{\milli\ohm} \\
        Gate-Ladung (0..\SI{10}{\volt}) & $Q_G$ & \SI{69}{\nano\coulomb} \\
        Eingangskapazität & $C_{iss}$ & \SI{4800}{\pico\farad} \\
        \bottomrule
    \end{tabular}
\end{table}

\begin{figure}[htbp]
    \centering
    \begin{tikzpicture}
        % Paths, nodes and wires:
        \node[shape=circle, draw, line width=0.5pt, minimum width=1.429cm] at (5.414, 9.268){};
        \node[nigfete, bodydiode] at (5.5, 9.27){};
        \node[shape=rectangle, minimum width=0.926cm, minimum height=0.544cm](N1) at (3.196, 9.013){} node[anchor=center] at (N1.text){$\mathrm{Gate}$};
        \node[shape=rectangle, minimum width=0.926cm, minimum height=0.544cm](N2) at (6.243, 8.135){} node[anchor=center] at (N2.text){$\mathrm{Source}$};
        \node[shape=rectangle, minimum width=0.926cm, minimum height=0.544cm](N3) at (6.176, 10.359){} node[anchor=center] at (N3.center){$\mathrm{Drain}$};
        \node[ocirc] at (3.952, 8.997){};
        \node[ocirc] at (5.502, 7.943){};
        \node[ocirc] at (5.501, 10.554){};
        \draw (5.5, 10.04) -| (5.5, 10.5);
        \draw (5.5, 8.5) -- (5.5, 8);
        \draw (4.5, 9) -- (4, 9);
    \end{tikzpicture}
    \caption{Symbol des verwendeten N-Kanal MOSFETs mit interner Freilaufdiode.}
    \label{fig:mosfet_symbol}
\end{figure}

\subsection{Berechnung der Verlustleistung}
Um die thermische Auslegung in Kapitel \ref{chap:thermik} vorzubereiten, werden die zu erwartenden Verluste im Nennbetrieb abgeschätzt. Die Gesamtverluste setzen sich aus den Durchlassverlusten ($P_{Cond}$) und den Schaltverlusten ($P_{SW}$) zusammen.

Die statischen Durchlassverluste lassen sich über das Ohmsche Gesetz herleiten. Bei einem angenommenen maximalen Phasenstrom von $I_{rms} = \SI{30}{\ampere}$ ergibt sich pro Schalter:
\begin{equation}
    P_{Cond} = I_{rms}^2 \cdot R_{DS(on)} = (\SI{30}{\ampere})^2 \cdot \SI{0,0034}{\ohm} = \SI{3,06}{\watt}
\end{equation}

Dies stellt den dominanten Anteil der Verluste dar. Hinzu kommen die Schaltverluste, die durch das nicht-ideale Schalten (gleichzeitiges Anliegen von Strom und Spannung während der Umschaltphasen) entstehen. Diese werden maßgeblich durch die Gate-Treiber-Stufe und die gewählte Schaltfrequenz beeinflusst.

\section{Auslegung der Gate-Treiber-Stufe}
Da N-Kanal-MOSFETs in der High-Side-Position einer Brückenschaltung eine Ansteuerspannung oberhalb des Versorgungspotentials benötigen ($V_G > V_S + V_{th}$), ist der Einsatz spezialisierter Halbbrückentreiber erforderlich. Für dieses Design wurde der \textbf{IR2104} ausgewählt.

Der Baustein übernimmt zwei zentrale Aufgaben: Zum einen generiert er mittels einer internen Logik die notwendige Totzeit (Deadtime) von typisch \SI{520}{\nano\second}, um ein gleichzeitiges Leiten beider Transistoren (Shoot-Through) hardwareseitig zu unterbinden. Zum anderen ermöglicht er über eine externe Bootstrap-Beschaltung die Versorgung des High-Side-Schalters.

\subsection{Dimensionierung der Bootstrap-Kapazität}
Ein kritischer Aspekt des Designs ist die Dimensionierung des Bootstrap-Kondensators ($C_{BS}$). Dieser Kondensator fungiert als schwimmende Spannungsquelle für den High-Side-Treiber. Er muss genügend Ladung $Q_{Total}$ speichern, um das Gate des MOSFETs vollständig umzuladen und Leckströme während der Leitphase zu kompensieren, ohne dass die Spannung signifikant einbricht.

Die benötigte Gesamtladung $Q_{Total}$ setzt sich additiv aus drei Komponenten zusammen: der Gateladung $Q_{Gate}$, dem Verbrauch des Level-Shifters $Q_{LS}$ und den Verlusten durch Leckströme $Q_{Leak}$.

\subsubsection*{1. Ermittlung der Gateladung}
Der Treiber wird mit einer Spannung von $V_{CC} = \SI{15}{\volt}$ betrieben. Da das Datenblatt des MOSFETs die Gateladung nur für \SI{10}{\volt} spezifiziert, muss der Wert extrapoliert werden. Unter Berücksichtigung der Eingangskapazität $C_{iss}$ ergibt sich eine effektive Ladungsmenge von:
\begin{equation}
    Q_{Gate} \approx Q_{G(10V)} + (V_{CC} - \SI{10}{\volt}) \cdot C_{iss} \approx \SI{111}{\nano\coulomb}
\end{equation}

\subsubsection*{2. Leckströme und Gesamtladung}
Zusätzlich muss der Kondensator die Leckströme der Schaltung während der maximalen Einschaltdauer ($t_{on}$) puffern. Für den Worst-Case ($f_s = \SI{20}{\kilo\hertz} \rightarrow t_{on} = \SI{50}{\micro\second}$) summieren sich die Leckströme von Treiber, Diode und Kondensator auf ca. \SI{115}{\micro\ampere}. Dies entspricht einem Ladungsverlust von $Q_{Leak} \approx \SI{5,8}{\nano\coulomb}$.

Mit einer Reserve für den Level-Shifter ($Q_{LS} \approx \SI{3}{\nano\coulomb}$) resultiert eine Gesamtladung von:
\begin{equation}
    Q_{Total} = \SI{111}{\nano\coulomb} + \SI{5,8}{\nano\coulomb} + \SI{3}{\nano\coulomb} \approx \SI{120}{\nano\coulomb}
\end{equation}

\subsubsection*{3. Ergebnis}
Um sicherzustellen, dass die Gate-Spannung auch am Ende der Einschaltdauer stabil bleibt, wird ein maximal zulässiger Spannungsabfall (Ripple) von $\Delta V_{BS} = \SI{1}{\volt}$ definiert. Die Mindestkapazität berechnet sich somit zu:
\begin{equation}
    C_{BS,min} = \frac{Q_{Total}}{\Delta V_{BS}} = \frac{\SI{120}{\nano\coulomb}}{\SI{1}{\volt}} = \SI{120}{\nano\farad}
\end{equation}

Um Toleranzen und Alterungseffekte (DC-Bias bei Keramikkondensatoren) auszugleichen, wird der nächsthöhere Normwert gewählt. Es kommt ein \textbf{\SI{220}{\nano\farad}} Keramikkondensator (X7R) zum Einsatz.

\section{Zwischenkreis-Auslegung}
Der Zwischenkreis (DC-Link) dient als Energiespeicher und stabilisiert die Spannung bei den hochfrequenten Schaltvorgängen.
\todo[inline]{Hier folgt später der Text zu den Elkos.}

\begin{figure}
    \centering
    %% Creator: Matplotlib, PGF backend
%%
%% To include the figure in your LaTeX document, write
%%   \input{<filename>.pgf}
%%
%% Make sure the required packages are loaded in your preamble
%%   \usepackage{pgf}
%%
%% Also ensure that all the required font packages are loaded; for instance,
%% the lmodern package is sometimes necessary when using math font.
%%   \usepackage{lmodern}
%%
%% Figures using additional raster images can only be included by \input if
%% they are in the same directory as the main LaTeX file. For loading figures
%% from other directories you can use the `import` package
%%   \usepackage{import}
%%
%% and then include the figures with
%%   \import{<path to file>}{<filename>.pgf}
%%
%% Matplotlib used the following preamble
%%   \def\mathdefault#1{#1}
%%   \everymath=\expandafter{\the\everymath\displaystyle}
%%   \IfFileExists{scrextend.sty}{
%%     \usepackage[fontsize=10.000000pt]{scrextend}
%%   }{
%%     \renewcommand{\normalsize}{\fontsize{10.000000}{12.000000}\selectfont}
%%     \normalsize
%%   }
%%   
%%   \ifdefined\pdftexversion\else  % non-pdftex case.
%%     \usepackage{fontspec}
%%     \setmainfont{DejaVuSerif.ttf}[Path=\detokenize{C:/Users/Pascal Lauer/AppData/Local/Packages/PythonSoftwareFoundation.Python.3.13_qbz5n2kfra8p0/LocalCache/local-packages/Python313/site-packages/matplotlib/mpl-data/fonts/ttf/}]
%%     \setsansfont{DejaVuSans.ttf}[Path=\detokenize{C:/Users/Pascal Lauer/AppData/Local/Packages/PythonSoftwareFoundation.Python.3.13_qbz5n2kfra8p0/LocalCache/local-packages/Python313/site-packages/matplotlib/mpl-data/fonts/ttf/}]
%%     \setmonofont{DejaVuSansMono.ttf}[Path=\detokenize{C:/Users/Pascal Lauer/AppData/Local/Packages/PythonSoftwareFoundation.Python.3.13_qbz5n2kfra8p0/LocalCache/local-packages/Python313/site-packages/matplotlib/mpl-data/fonts/ttf/}]
%%   \fi
%%   \makeatletter\@ifpackageloaded{underscore}{}{\usepackage[strings]{underscore}}\makeatother
%%
\begingroup%
\makeatletter%
\begin{pgfpicture}%
\pgfpathrectangle{\pgfpointorigin}{\pgfqpoint{6.000000in}{6.000000in}}%
\pgfusepath{use as bounding box, clip}%
\begin{pgfscope}%
\pgfsetbuttcap%
\pgfsetmiterjoin%
\definecolor{currentfill}{rgb}{1.000000,1.000000,1.000000}%
\pgfsetfillcolor{currentfill}%
\pgfsetlinewidth{0.000000pt}%
\definecolor{currentstroke}{rgb}{1.000000,1.000000,1.000000}%
\pgfsetstrokecolor{currentstroke}%
\pgfsetdash{}{0pt}%
\pgfpathmoveto{\pgfqpoint{0.000000in}{0.000000in}}%
\pgfpathlineto{\pgfqpoint{6.000000in}{0.000000in}}%
\pgfpathlineto{\pgfqpoint{6.000000in}{6.000000in}}%
\pgfpathlineto{\pgfqpoint{0.000000in}{6.000000in}}%
\pgfpathlineto{\pgfqpoint{0.000000in}{0.000000in}}%
\pgfpathclose%
\pgfusepath{fill}%
\end{pgfscope}%
\begin{pgfscope}%
\pgfsetbuttcap%
\pgfsetmiterjoin%
\definecolor{currentfill}{rgb}{1.000000,1.000000,1.000000}%
\pgfsetfillcolor{currentfill}%
\pgfsetlinewidth{0.000000pt}%
\definecolor{currentstroke}{rgb}{0.000000,0.000000,0.000000}%
\pgfsetstrokecolor{currentstroke}%
\pgfsetstrokeopacity{0.000000}%
\pgfsetdash{}{0pt}%
\pgfpathmoveto{\pgfqpoint{0.953437in}{0.609444in}}%
\pgfpathlineto{\pgfqpoint{5.792674in}{0.609444in}}%
\pgfpathlineto{\pgfqpoint{5.792674in}{5.448681in}}%
\pgfpathlineto{\pgfqpoint{0.953437in}{5.448681in}}%
\pgfpathlineto{\pgfqpoint{0.953437in}{0.609444in}}%
\pgfpathclose%
\pgfusepath{fill}%
\end{pgfscope}%
\begin{pgfscope}%
\pgfpathrectangle{\pgfqpoint{0.953437in}{0.609444in}}{\pgfqpoint{4.839236in}{4.839236in}}%
\pgfusepath{clip}%
\pgfsetbuttcap%
\pgfsetmiterjoin%
\pgfsetlinewidth{1.003750pt}%
\definecolor{currentstroke}{rgb}{0.203922,0.596078,0.858824}%
\pgfsetstrokecolor{currentstroke}%
\pgfsetstrokeopacity{0.600000}%
\pgfsetdash{{3.700000pt}{1.600000pt}}{0.000000pt}%
\pgfpathmoveto{\pgfqpoint{3.373056in}{1.532312in}}%
\pgfpathcurveto{\pgfqpoint{3.769998in}{1.532312in}}{\pgfqpoint{4.150737in}{1.690019in}}{\pgfqpoint{4.431418in}{1.970700in}}%
\pgfpathcurveto{\pgfqpoint{4.712099in}{2.251381in}}{\pgfqpoint{4.869806in}{2.632120in}}{\pgfqpoint{4.869806in}{3.029063in}}%
\pgfpathcurveto{\pgfqpoint{4.869806in}{3.426005in}}{\pgfqpoint{4.712099in}{3.806744in}}{\pgfqpoint{4.431418in}{4.087425in}}%
\pgfpathcurveto{\pgfqpoint{4.150737in}{4.368106in}}{\pgfqpoint{3.769998in}{4.525813in}}{\pgfqpoint{3.373056in}{4.525813in}}%
\pgfpathcurveto{\pgfqpoint{2.976113in}{4.525813in}}{\pgfqpoint{2.595374in}{4.368106in}}{\pgfqpoint{2.314693in}{4.087425in}}%
\pgfpathcurveto{\pgfqpoint{2.034012in}{3.806744in}}{\pgfqpoint{1.876305in}{3.426005in}}{\pgfqpoint{1.876305in}{3.029063in}}%
\pgfpathcurveto{\pgfqpoint{1.876305in}{2.632120in}}{\pgfqpoint{2.034012in}{2.251381in}}{\pgfqpoint{2.314693in}{1.970700in}}%
\pgfpathcurveto{\pgfqpoint{2.595374in}{1.690019in}}{\pgfqpoint{2.976113in}{1.532312in}}{\pgfqpoint{3.373056in}{1.532312in}}%
\pgfpathlineto{\pgfqpoint{3.373056in}{1.532312in}}%
\pgfpathclose%
\pgfusepath{stroke}%
\end{pgfscope}%
\begin{pgfscope}%
\pgfpathrectangle{\pgfqpoint{0.953437in}{0.609444in}}{\pgfqpoint{4.839236in}{4.839236in}}%
\pgfusepath{clip}%
\pgfsetbuttcap%
\pgfsetroundjoin%
\pgfsetlinewidth{0.803000pt}%
\definecolor{currentstroke}{rgb}{0.690196,0.690196,0.690196}%
\pgfsetstrokecolor{currentstroke}%
\pgfsetstrokeopacity{0.300000}%
\pgfsetdash{{0.800000pt}{1.320000pt}}{0.000000pt}%
\pgfpathmoveto{\pgfqpoint{1.428720in}{0.609444in}}%
\pgfpathlineto{\pgfqpoint{1.428720in}{5.448681in}}%
\pgfusepath{stroke}%
\end{pgfscope}%
\begin{pgfscope}%
\pgfsetbuttcap%
\pgfsetroundjoin%
\definecolor{currentfill}{rgb}{0.000000,0.000000,0.000000}%
\pgfsetfillcolor{currentfill}%
\pgfsetlinewidth{0.803000pt}%
\definecolor{currentstroke}{rgb}{0.000000,0.000000,0.000000}%
\pgfsetstrokecolor{currentstroke}%
\pgfsetdash{}{0pt}%
\pgfsys@defobject{currentmarker}{\pgfqpoint{0.000000in}{-0.048611in}}{\pgfqpoint{0.000000in}{0.000000in}}{%
\pgfpathmoveto{\pgfqpoint{0.000000in}{0.000000in}}%
\pgfpathlineto{\pgfqpoint{0.000000in}{-0.048611in}}%
\pgfusepath{stroke,fill}%
}%
\begin{pgfscope}%
\pgfsys@transformshift{1.428720in}{0.609444in}%
\pgfsys@useobject{currentmarker}{}%
\end{pgfscope}%
\end{pgfscope}%
\begin{pgfscope}%
\definecolor{textcolor}{rgb}{0.000000,0.000000,0.000000}%
\pgfsetstrokecolor{textcolor}%
\pgfsetfillcolor{textcolor}%
\pgftext[x=1.428720in,y=0.512222in,,top]{\color{textcolor}{\rmfamily\fontsize{10.000000}{12.000000}\selectfont\catcode`\^=\active\def^{\ifmmode\sp\else\^{}\fi}\catcode`\%=\active\def%{\%}\ensuremath{-}0.75}}%
\end{pgfscope}%
\begin{pgfscope}%
\pgfpathrectangle{\pgfqpoint{0.953437in}{0.609444in}}{\pgfqpoint{4.839236in}{4.839236in}}%
\pgfusepath{clip}%
\pgfsetbuttcap%
\pgfsetroundjoin%
\pgfsetlinewidth{0.803000pt}%
\definecolor{currentstroke}{rgb}{0.690196,0.690196,0.690196}%
\pgfsetstrokecolor{currentstroke}%
\pgfsetstrokeopacity{0.300000}%
\pgfsetdash{{0.800000pt}{1.320000pt}}{0.000000pt}%
\pgfpathmoveto{\pgfqpoint{2.076832in}{0.609444in}}%
\pgfpathlineto{\pgfqpoint{2.076832in}{5.448681in}}%
\pgfusepath{stroke}%
\end{pgfscope}%
\begin{pgfscope}%
\pgfsetbuttcap%
\pgfsetroundjoin%
\definecolor{currentfill}{rgb}{0.000000,0.000000,0.000000}%
\pgfsetfillcolor{currentfill}%
\pgfsetlinewidth{0.803000pt}%
\definecolor{currentstroke}{rgb}{0.000000,0.000000,0.000000}%
\pgfsetstrokecolor{currentstroke}%
\pgfsetdash{}{0pt}%
\pgfsys@defobject{currentmarker}{\pgfqpoint{0.000000in}{-0.048611in}}{\pgfqpoint{0.000000in}{0.000000in}}{%
\pgfpathmoveto{\pgfqpoint{0.000000in}{0.000000in}}%
\pgfpathlineto{\pgfqpoint{0.000000in}{-0.048611in}}%
\pgfusepath{stroke,fill}%
}%
\begin{pgfscope}%
\pgfsys@transformshift{2.076832in}{0.609444in}%
\pgfsys@useobject{currentmarker}{}%
\end{pgfscope}%
\end{pgfscope}%
\begin{pgfscope}%
\definecolor{textcolor}{rgb}{0.000000,0.000000,0.000000}%
\pgfsetstrokecolor{textcolor}%
\pgfsetfillcolor{textcolor}%
\pgftext[x=2.076832in,y=0.512222in,,top]{\color{textcolor}{\rmfamily\fontsize{10.000000}{12.000000}\selectfont\catcode`\^=\active\def^{\ifmmode\sp\else\^{}\fi}\catcode`\%=\active\def%{\%}\ensuremath{-}0.50}}%
\end{pgfscope}%
\begin{pgfscope}%
\pgfpathrectangle{\pgfqpoint{0.953437in}{0.609444in}}{\pgfqpoint{4.839236in}{4.839236in}}%
\pgfusepath{clip}%
\pgfsetbuttcap%
\pgfsetroundjoin%
\pgfsetlinewidth{0.803000pt}%
\definecolor{currentstroke}{rgb}{0.690196,0.690196,0.690196}%
\pgfsetstrokecolor{currentstroke}%
\pgfsetstrokeopacity{0.300000}%
\pgfsetdash{{0.800000pt}{1.320000pt}}{0.000000pt}%
\pgfpathmoveto{\pgfqpoint{2.724944in}{0.609444in}}%
\pgfpathlineto{\pgfqpoint{2.724944in}{5.448681in}}%
\pgfusepath{stroke}%
\end{pgfscope}%
\begin{pgfscope}%
\pgfsetbuttcap%
\pgfsetroundjoin%
\definecolor{currentfill}{rgb}{0.000000,0.000000,0.000000}%
\pgfsetfillcolor{currentfill}%
\pgfsetlinewidth{0.803000pt}%
\definecolor{currentstroke}{rgb}{0.000000,0.000000,0.000000}%
\pgfsetstrokecolor{currentstroke}%
\pgfsetdash{}{0pt}%
\pgfsys@defobject{currentmarker}{\pgfqpoint{0.000000in}{-0.048611in}}{\pgfqpoint{0.000000in}{0.000000in}}{%
\pgfpathmoveto{\pgfqpoint{0.000000in}{0.000000in}}%
\pgfpathlineto{\pgfqpoint{0.000000in}{-0.048611in}}%
\pgfusepath{stroke,fill}%
}%
\begin{pgfscope}%
\pgfsys@transformshift{2.724944in}{0.609444in}%
\pgfsys@useobject{currentmarker}{}%
\end{pgfscope}%
\end{pgfscope}%
\begin{pgfscope}%
\definecolor{textcolor}{rgb}{0.000000,0.000000,0.000000}%
\pgfsetstrokecolor{textcolor}%
\pgfsetfillcolor{textcolor}%
\pgftext[x=2.724944in,y=0.512222in,,top]{\color{textcolor}{\rmfamily\fontsize{10.000000}{12.000000}\selectfont\catcode`\^=\active\def^{\ifmmode\sp\else\^{}\fi}\catcode`\%=\active\def%{\%}\ensuremath{-}0.25}}%
\end{pgfscope}%
\begin{pgfscope}%
\pgfpathrectangle{\pgfqpoint{0.953437in}{0.609444in}}{\pgfqpoint{4.839236in}{4.839236in}}%
\pgfusepath{clip}%
\pgfsetbuttcap%
\pgfsetroundjoin%
\pgfsetlinewidth{0.803000pt}%
\definecolor{currentstroke}{rgb}{0.690196,0.690196,0.690196}%
\pgfsetstrokecolor{currentstroke}%
\pgfsetstrokeopacity{0.300000}%
\pgfsetdash{{0.800000pt}{1.320000pt}}{0.000000pt}%
\pgfpathmoveto{\pgfqpoint{3.373056in}{0.609444in}}%
\pgfpathlineto{\pgfqpoint{3.373056in}{5.448681in}}%
\pgfusepath{stroke}%
\end{pgfscope}%
\begin{pgfscope}%
\pgfsetbuttcap%
\pgfsetroundjoin%
\definecolor{currentfill}{rgb}{0.000000,0.000000,0.000000}%
\pgfsetfillcolor{currentfill}%
\pgfsetlinewidth{0.803000pt}%
\definecolor{currentstroke}{rgb}{0.000000,0.000000,0.000000}%
\pgfsetstrokecolor{currentstroke}%
\pgfsetdash{}{0pt}%
\pgfsys@defobject{currentmarker}{\pgfqpoint{0.000000in}{-0.048611in}}{\pgfqpoint{0.000000in}{0.000000in}}{%
\pgfpathmoveto{\pgfqpoint{0.000000in}{0.000000in}}%
\pgfpathlineto{\pgfqpoint{0.000000in}{-0.048611in}}%
\pgfusepath{stroke,fill}%
}%
\begin{pgfscope}%
\pgfsys@transformshift{3.373056in}{0.609444in}%
\pgfsys@useobject{currentmarker}{}%
\end{pgfscope}%
\end{pgfscope}%
\begin{pgfscope}%
\definecolor{textcolor}{rgb}{0.000000,0.000000,0.000000}%
\pgfsetstrokecolor{textcolor}%
\pgfsetfillcolor{textcolor}%
\pgftext[x=3.373056in,y=0.512222in,,top]{\color{textcolor}{\rmfamily\fontsize{10.000000}{12.000000}\selectfont\catcode`\^=\active\def^{\ifmmode\sp\else\^{}\fi}\catcode`\%=\active\def%{\%}0.00}}%
\end{pgfscope}%
\begin{pgfscope}%
\pgfpathrectangle{\pgfqpoint{0.953437in}{0.609444in}}{\pgfqpoint{4.839236in}{4.839236in}}%
\pgfusepath{clip}%
\pgfsetbuttcap%
\pgfsetroundjoin%
\pgfsetlinewidth{0.803000pt}%
\definecolor{currentstroke}{rgb}{0.690196,0.690196,0.690196}%
\pgfsetstrokecolor{currentstroke}%
\pgfsetstrokeopacity{0.300000}%
\pgfsetdash{{0.800000pt}{1.320000pt}}{0.000000pt}%
\pgfpathmoveto{\pgfqpoint{4.021168in}{0.609444in}}%
\pgfpathlineto{\pgfqpoint{4.021168in}{5.448681in}}%
\pgfusepath{stroke}%
\end{pgfscope}%
\begin{pgfscope}%
\pgfsetbuttcap%
\pgfsetroundjoin%
\definecolor{currentfill}{rgb}{0.000000,0.000000,0.000000}%
\pgfsetfillcolor{currentfill}%
\pgfsetlinewidth{0.803000pt}%
\definecolor{currentstroke}{rgb}{0.000000,0.000000,0.000000}%
\pgfsetstrokecolor{currentstroke}%
\pgfsetdash{}{0pt}%
\pgfsys@defobject{currentmarker}{\pgfqpoint{0.000000in}{-0.048611in}}{\pgfqpoint{0.000000in}{0.000000in}}{%
\pgfpathmoveto{\pgfqpoint{0.000000in}{0.000000in}}%
\pgfpathlineto{\pgfqpoint{0.000000in}{-0.048611in}}%
\pgfusepath{stroke,fill}%
}%
\begin{pgfscope}%
\pgfsys@transformshift{4.021168in}{0.609444in}%
\pgfsys@useobject{currentmarker}{}%
\end{pgfscope}%
\end{pgfscope}%
\begin{pgfscope}%
\definecolor{textcolor}{rgb}{0.000000,0.000000,0.000000}%
\pgfsetstrokecolor{textcolor}%
\pgfsetfillcolor{textcolor}%
\pgftext[x=4.021168in,y=0.512222in,,top]{\color{textcolor}{\rmfamily\fontsize{10.000000}{12.000000}\selectfont\catcode`\^=\active\def^{\ifmmode\sp\else\^{}\fi}\catcode`\%=\active\def%{\%}0.25}}%
\end{pgfscope}%
\begin{pgfscope}%
\pgfpathrectangle{\pgfqpoint{0.953437in}{0.609444in}}{\pgfqpoint{4.839236in}{4.839236in}}%
\pgfusepath{clip}%
\pgfsetbuttcap%
\pgfsetroundjoin%
\pgfsetlinewidth{0.803000pt}%
\definecolor{currentstroke}{rgb}{0.690196,0.690196,0.690196}%
\pgfsetstrokecolor{currentstroke}%
\pgfsetstrokeopacity{0.300000}%
\pgfsetdash{{0.800000pt}{1.320000pt}}{0.000000pt}%
\pgfpathmoveto{\pgfqpoint{4.669280in}{0.609444in}}%
\pgfpathlineto{\pgfqpoint{4.669280in}{5.448681in}}%
\pgfusepath{stroke}%
\end{pgfscope}%
\begin{pgfscope}%
\pgfsetbuttcap%
\pgfsetroundjoin%
\definecolor{currentfill}{rgb}{0.000000,0.000000,0.000000}%
\pgfsetfillcolor{currentfill}%
\pgfsetlinewidth{0.803000pt}%
\definecolor{currentstroke}{rgb}{0.000000,0.000000,0.000000}%
\pgfsetstrokecolor{currentstroke}%
\pgfsetdash{}{0pt}%
\pgfsys@defobject{currentmarker}{\pgfqpoint{0.000000in}{-0.048611in}}{\pgfqpoint{0.000000in}{0.000000in}}{%
\pgfpathmoveto{\pgfqpoint{0.000000in}{0.000000in}}%
\pgfpathlineto{\pgfqpoint{0.000000in}{-0.048611in}}%
\pgfusepath{stroke,fill}%
}%
\begin{pgfscope}%
\pgfsys@transformshift{4.669280in}{0.609444in}%
\pgfsys@useobject{currentmarker}{}%
\end{pgfscope}%
\end{pgfscope}%
\begin{pgfscope}%
\definecolor{textcolor}{rgb}{0.000000,0.000000,0.000000}%
\pgfsetstrokecolor{textcolor}%
\pgfsetfillcolor{textcolor}%
\pgftext[x=4.669280in,y=0.512222in,,top]{\color{textcolor}{\rmfamily\fontsize{10.000000}{12.000000}\selectfont\catcode`\^=\active\def^{\ifmmode\sp\else\^{}\fi}\catcode`\%=\active\def%{\%}0.50}}%
\end{pgfscope}%
\begin{pgfscope}%
\pgfpathrectangle{\pgfqpoint{0.953437in}{0.609444in}}{\pgfqpoint{4.839236in}{4.839236in}}%
\pgfusepath{clip}%
\pgfsetbuttcap%
\pgfsetroundjoin%
\pgfsetlinewidth{0.803000pt}%
\definecolor{currentstroke}{rgb}{0.690196,0.690196,0.690196}%
\pgfsetstrokecolor{currentstroke}%
\pgfsetstrokeopacity{0.300000}%
\pgfsetdash{{0.800000pt}{1.320000pt}}{0.000000pt}%
\pgfpathmoveto{\pgfqpoint{5.317391in}{0.609444in}}%
\pgfpathlineto{\pgfqpoint{5.317391in}{5.448681in}}%
\pgfusepath{stroke}%
\end{pgfscope}%
\begin{pgfscope}%
\pgfsetbuttcap%
\pgfsetroundjoin%
\definecolor{currentfill}{rgb}{0.000000,0.000000,0.000000}%
\pgfsetfillcolor{currentfill}%
\pgfsetlinewidth{0.803000pt}%
\definecolor{currentstroke}{rgb}{0.000000,0.000000,0.000000}%
\pgfsetstrokecolor{currentstroke}%
\pgfsetdash{}{0pt}%
\pgfsys@defobject{currentmarker}{\pgfqpoint{0.000000in}{-0.048611in}}{\pgfqpoint{0.000000in}{0.000000in}}{%
\pgfpathmoveto{\pgfqpoint{0.000000in}{0.000000in}}%
\pgfpathlineto{\pgfqpoint{0.000000in}{-0.048611in}}%
\pgfusepath{stroke,fill}%
}%
\begin{pgfscope}%
\pgfsys@transformshift{5.317391in}{0.609444in}%
\pgfsys@useobject{currentmarker}{}%
\end{pgfscope}%
\end{pgfscope}%
\begin{pgfscope}%
\definecolor{textcolor}{rgb}{0.000000,0.000000,0.000000}%
\pgfsetstrokecolor{textcolor}%
\pgfsetfillcolor{textcolor}%
\pgftext[x=5.317391in,y=0.512222in,,top]{\color{textcolor}{\rmfamily\fontsize{10.000000}{12.000000}\selectfont\catcode`\^=\active\def^{\ifmmode\sp\else\^{}\fi}\catcode`\%=\active\def%{\%}0.75}}%
\end{pgfscope}%
\begin{pgfscope}%
\definecolor{textcolor}{rgb}{0.000000,0.000000,0.000000}%
\pgfsetstrokecolor{textcolor}%
\pgfsetfillcolor{textcolor}%
\pgftext[x=3.373056in,y=0.322254in,,top]{\color{textcolor}{\rmfamily\fontsize{11.000000}{13.200000}\selectfont\catcode`\^=\active\def^{\ifmmode\sp\else\^{}\fi}\catcode`\%=\active\def%{\%}$\alpha$ (U-Alpha)}}%
\end{pgfscope}%
\begin{pgfscope}%
\pgfpathrectangle{\pgfqpoint{0.953437in}{0.609444in}}{\pgfqpoint{4.839236in}{4.839236in}}%
\pgfusepath{clip}%
\pgfsetbuttcap%
\pgfsetroundjoin%
\pgfsetlinewidth{0.803000pt}%
\definecolor{currentstroke}{rgb}{0.690196,0.690196,0.690196}%
\pgfsetstrokecolor{currentstroke}%
\pgfsetstrokeopacity{0.300000}%
\pgfsetdash{{0.800000pt}{1.320000pt}}{0.000000pt}%
\pgfpathmoveto{\pgfqpoint{0.953437in}{1.084727in}}%
\pgfpathlineto{\pgfqpoint{5.792674in}{1.084727in}}%
\pgfusepath{stroke}%
\end{pgfscope}%
\begin{pgfscope}%
\pgfsetbuttcap%
\pgfsetroundjoin%
\definecolor{currentfill}{rgb}{0.000000,0.000000,0.000000}%
\pgfsetfillcolor{currentfill}%
\pgfsetlinewidth{0.803000pt}%
\definecolor{currentstroke}{rgb}{0.000000,0.000000,0.000000}%
\pgfsetstrokecolor{currentstroke}%
\pgfsetdash{}{0pt}%
\pgfsys@defobject{currentmarker}{\pgfqpoint{-0.048611in}{0.000000in}}{\pgfqpoint{-0.000000in}{0.000000in}}{%
\pgfpathmoveto{\pgfqpoint{-0.000000in}{0.000000in}}%
\pgfpathlineto{\pgfqpoint{-0.048611in}{0.000000in}}%
\pgfusepath{stroke,fill}%
}%
\begin{pgfscope}%
\pgfsys@transformshift{0.953437in}{1.084727in}%
\pgfsys@useobject{currentmarker}{}%
\end{pgfscope}%
\end{pgfscope}%
\begin{pgfscope}%
\definecolor{textcolor}{rgb}{0.000000,0.000000,0.000000}%
\pgfsetstrokecolor{textcolor}%
\pgfsetfillcolor{textcolor}%
\pgftext[x=0.438945in, y=1.031965in, left, base]{\color{textcolor}{\rmfamily\fontsize{10.000000}{12.000000}\selectfont\catcode`\^=\active\def^{\ifmmode\sp\else\^{}\fi}\catcode`\%=\active\def%{\%}\ensuremath{-}0.75}}%
\end{pgfscope}%
\begin{pgfscope}%
\pgfpathrectangle{\pgfqpoint{0.953437in}{0.609444in}}{\pgfqpoint{4.839236in}{4.839236in}}%
\pgfusepath{clip}%
\pgfsetbuttcap%
\pgfsetroundjoin%
\pgfsetlinewidth{0.803000pt}%
\definecolor{currentstroke}{rgb}{0.690196,0.690196,0.690196}%
\pgfsetstrokecolor{currentstroke}%
\pgfsetstrokeopacity{0.300000}%
\pgfsetdash{{0.800000pt}{1.320000pt}}{0.000000pt}%
\pgfpathmoveto{\pgfqpoint{0.953437in}{1.732839in}}%
\pgfpathlineto{\pgfqpoint{5.792674in}{1.732839in}}%
\pgfusepath{stroke}%
\end{pgfscope}%
\begin{pgfscope}%
\pgfsetbuttcap%
\pgfsetroundjoin%
\definecolor{currentfill}{rgb}{0.000000,0.000000,0.000000}%
\pgfsetfillcolor{currentfill}%
\pgfsetlinewidth{0.803000pt}%
\definecolor{currentstroke}{rgb}{0.000000,0.000000,0.000000}%
\pgfsetstrokecolor{currentstroke}%
\pgfsetdash{}{0pt}%
\pgfsys@defobject{currentmarker}{\pgfqpoint{-0.048611in}{0.000000in}}{\pgfqpoint{-0.000000in}{0.000000in}}{%
\pgfpathmoveto{\pgfqpoint{-0.000000in}{0.000000in}}%
\pgfpathlineto{\pgfqpoint{-0.048611in}{0.000000in}}%
\pgfusepath{stroke,fill}%
}%
\begin{pgfscope}%
\pgfsys@transformshift{0.953437in}{1.732839in}%
\pgfsys@useobject{currentmarker}{}%
\end{pgfscope}%
\end{pgfscope}%
\begin{pgfscope}%
\definecolor{textcolor}{rgb}{0.000000,0.000000,0.000000}%
\pgfsetstrokecolor{textcolor}%
\pgfsetfillcolor{textcolor}%
\pgftext[x=0.438945in, y=1.680077in, left, base]{\color{textcolor}{\rmfamily\fontsize{10.000000}{12.000000}\selectfont\catcode`\^=\active\def^{\ifmmode\sp\else\^{}\fi}\catcode`\%=\active\def%{\%}\ensuremath{-}0.50}}%
\end{pgfscope}%
\begin{pgfscope}%
\pgfpathrectangle{\pgfqpoint{0.953437in}{0.609444in}}{\pgfqpoint{4.839236in}{4.839236in}}%
\pgfusepath{clip}%
\pgfsetbuttcap%
\pgfsetroundjoin%
\pgfsetlinewidth{0.803000pt}%
\definecolor{currentstroke}{rgb}{0.690196,0.690196,0.690196}%
\pgfsetstrokecolor{currentstroke}%
\pgfsetstrokeopacity{0.300000}%
\pgfsetdash{{0.800000pt}{1.320000pt}}{0.000000pt}%
\pgfpathmoveto{\pgfqpoint{0.953437in}{2.380951in}}%
\pgfpathlineto{\pgfqpoint{5.792674in}{2.380951in}}%
\pgfusepath{stroke}%
\end{pgfscope}%
\begin{pgfscope}%
\pgfsetbuttcap%
\pgfsetroundjoin%
\definecolor{currentfill}{rgb}{0.000000,0.000000,0.000000}%
\pgfsetfillcolor{currentfill}%
\pgfsetlinewidth{0.803000pt}%
\definecolor{currentstroke}{rgb}{0.000000,0.000000,0.000000}%
\pgfsetstrokecolor{currentstroke}%
\pgfsetdash{}{0pt}%
\pgfsys@defobject{currentmarker}{\pgfqpoint{-0.048611in}{0.000000in}}{\pgfqpoint{-0.000000in}{0.000000in}}{%
\pgfpathmoveto{\pgfqpoint{-0.000000in}{0.000000in}}%
\pgfpathlineto{\pgfqpoint{-0.048611in}{0.000000in}}%
\pgfusepath{stroke,fill}%
}%
\begin{pgfscope}%
\pgfsys@transformshift{0.953437in}{2.380951in}%
\pgfsys@useobject{currentmarker}{}%
\end{pgfscope}%
\end{pgfscope}%
\begin{pgfscope}%
\definecolor{textcolor}{rgb}{0.000000,0.000000,0.000000}%
\pgfsetstrokecolor{textcolor}%
\pgfsetfillcolor{textcolor}%
\pgftext[x=0.438945in, y=2.328189in, left, base]{\color{textcolor}{\rmfamily\fontsize{10.000000}{12.000000}\selectfont\catcode`\^=\active\def^{\ifmmode\sp\else\^{}\fi}\catcode`\%=\active\def%{\%}\ensuremath{-}0.25}}%
\end{pgfscope}%
\begin{pgfscope}%
\pgfpathrectangle{\pgfqpoint{0.953437in}{0.609444in}}{\pgfqpoint{4.839236in}{4.839236in}}%
\pgfusepath{clip}%
\pgfsetbuttcap%
\pgfsetroundjoin%
\pgfsetlinewidth{0.803000pt}%
\definecolor{currentstroke}{rgb}{0.690196,0.690196,0.690196}%
\pgfsetstrokecolor{currentstroke}%
\pgfsetstrokeopacity{0.300000}%
\pgfsetdash{{0.800000pt}{1.320000pt}}{0.000000pt}%
\pgfpathmoveto{\pgfqpoint{0.953437in}{3.029063in}}%
\pgfpathlineto{\pgfqpoint{5.792674in}{3.029063in}}%
\pgfusepath{stroke}%
\end{pgfscope}%
\begin{pgfscope}%
\pgfsetbuttcap%
\pgfsetroundjoin%
\definecolor{currentfill}{rgb}{0.000000,0.000000,0.000000}%
\pgfsetfillcolor{currentfill}%
\pgfsetlinewidth{0.803000pt}%
\definecolor{currentstroke}{rgb}{0.000000,0.000000,0.000000}%
\pgfsetstrokecolor{currentstroke}%
\pgfsetdash{}{0pt}%
\pgfsys@defobject{currentmarker}{\pgfqpoint{-0.048611in}{0.000000in}}{\pgfqpoint{-0.000000in}{0.000000in}}{%
\pgfpathmoveto{\pgfqpoint{-0.000000in}{0.000000in}}%
\pgfpathlineto{\pgfqpoint{-0.048611in}{0.000000in}}%
\pgfusepath{stroke,fill}%
}%
\begin{pgfscope}%
\pgfsys@transformshift{0.953437in}{3.029063in}%
\pgfsys@useobject{currentmarker}{}%
\end{pgfscope}%
\end{pgfscope}%
\begin{pgfscope}%
\definecolor{textcolor}{rgb}{0.000000,0.000000,0.000000}%
\pgfsetstrokecolor{textcolor}%
\pgfsetfillcolor{textcolor}%
\pgftext[x=0.546970in, y=2.976301in, left, base]{\color{textcolor}{\rmfamily\fontsize{10.000000}{12.000000}\selectfont\catcode`\^=\active\def^{\ifmmode\sp\else\^{}\fi}\catcode`\%=\active\def%{\%}0.00}}%
\end{pgfscope}%
\begin{pgfscope}%
\pgfpathrectangle{\pgfqpoint{0.953437in}{0.609444in}}{\pgfqpoint{4.839236in}{4.839236in}}%
\pgfusepath{clip}%
\pgfsetbuttcap%
\pgfsetroundjoin%
\pgfsetlinewidth{0.803000pt}%
\definecolor{currentstroke}{rgb}{0.690196,0.690196,0.690196}%
\pgfsetstrokecolor{currentstroke}%
\pgfsetstrokeopacity{0.300000}%
\pgfsetdash{{0.800000pt}{1.320000pt}}{0.000000pt}%
\pgfpathmoveto{\pgfqpoint{0.953437in}{3.677174in}}%
\pgfpathlineto{\pgfqpoint{5.792674in}{3.677174in}}%
\pgfusepath{stroke}%
\end{pgfscope}%
\begin{pgfscope}%
\pgfsetbuttcap%
\pgfsetroundjoin%
\definecolor{currentfill}{rgb}{0.000000,0.000000,0.000000}%
\pgfsetfillcolor{currentfill}%
\pgfsetlinewidth{0.803000pt}%
\definecolor{currentstroke}{rgb}{0.000000,0.000000,0.000000}%
\pgfsetstrokecolor{currentstroke}%
\pgfsetdash{}{0pt}%
\pgfsys@defobject{currentmarker}{\pgfqpoint{-0.048611in}{0.000000in}}{\pgfqpoint{-0.000000in}{0.000000in}}{%
\pgfpathmoveto{\pgfqpoint{-0.000000in}{0.000000in}}%
\pgfpathlineto{\pgfqpoint{-0.048611in}{0.000000in}}%
\pgfusepath{stroke,fill}%
}%
\begin{pgfscope}%
\pgfsys@transformshift{0.953437in}{3.677174in}%
\pgfsys@useobject{currentmarker}{}%
\end{pgfscope}%
\end{pgfscope}%
\begin{pgfscope}%
\definecolor{textcolor}{rgb}{0.000000,0.000000,0.000000}%
\pgfsetstrokecolor{textcolor}%
\pgfsetfillcolor{textcolor}%
\pgftext[x=0.546970in, y=3.624413in, left, base]{\color{textcolor}{\rmfamily\fontsize{10.000000}{12.000000}\selectfont\catcode`\^=\active\def^{\ifmmode\sp\else\^{}\fi}\catcode`\%=\active\def%{\%}0.25}}%
\end{pgfscope}%
\begin{pgfscope}%
\pgfpathrectangle{\pgfqpoint{0.953437in}{0.609444in}}{\pgfqpoint{4.839236in}{4.839236in}}%
\pgfusepath{clip}%
\pgfsetbuttcap%
\pgfsetroundjoin%
\pgfsetlinewidth{0.803000pt}%
\definecolor{currentstroke}{rgb}{0.690196,0.690196,0.690196}%
\pgfsetstrokecolor{currentstroke}%
\pgfsetstrokeopacity{0.300000}%
\pgfsetdash{{0.800000pt}{1.320000pt}}{0.000000pt}%
\pgfpathmoveto{\pgfqpoint{0.953437in}{4.325286in}}%
\pgfpathlineto{\pgfqpoint{5.792674in}{4.325286in}}%
\pgfusepath{stroke}%
\end{pgfscope}%
\begin{pgfscope}%
\pgfsetbuttcap%
\pgfsetroundjoin%
\definecolor{currentfill}{rgb}{0.000000,0.000000,0.000000}%
\pgfsetfillcolor{currentfill}%
\pgfsetlinewidth{0.803000pt}%
\definecolor{currentstroke}{rgb}{0.000000,0.000000,0.000000}%
\pgfsetstrokecolor{currentstroke}%
\pgfsetdash{}{0pt}%
\pgfsys@defobject{currentmarker}{\pgfqpoint{-0.048611in}{0.000000in}}{\pgfqpoint{-0.000000in}{0.000000in}}{%
\pgfpathmoveto{\pgfqpoint{-0.000000in}{0.000000in}}%
\pgfpathlineto{\pgfqpoint{-0.048611in}{0.000000in}}%
\pgfusepath{stroke,fill}%
}%
\begin{pgfscope}%
\pgfsys@transformshift{0.953437in}{4.325286in}%
\pgfsys@useobject{currentmarker}{}%
\end{pgfscope}%
\end{pgfscope}%
\begin{pgfscope}%
\definecolor{textcolor}{rgb}{0.000000,0.000000,0.000000}%
\pgfsetstrokecolor{textcolor}%
\pgfsetfillcolor{textcolor}%
\pgftext[x=0.546970in, y=4.272525in, left, base]{\color{textcolor}{\rmfamily\fontsize{10.000000}{12.000000}\selectfont\catcode`\^=\active\def^{\ifmmode\sp\else\^{}\fi}\catcode`\%=\active\def%{\%}0.50}}%
\end{pgfscope}%
\begin{pgfscope}%
\pgfpathrectangle{\pgfqpoint{0.953437in}{0.609444in}}{\pgfqpoint{4.839236in}{4.839236in}}%
\pgfusepath{clip}%
\pgfsetbuttcap%
\pgfsetroundjoin%
\pgfsetlinewidth{0.803000pt}%
\definecolor{currentstroke}{rgb}{0.690196,0.690196,0.690196}%
\pgfsetstrokecolor{currentstroke}%
\pgfsetstrokeopacity{0.300000}%
\pgfsetdash{{0.800000pt}{1.320000pt}}{0.000000pt}%
\pgfpathmoveto{\pgfqpoint{0.953437in}{4.973398in}}%
\pgfpathlineto{\pgfqpoint{5.792674in}{4.973398in}}%
\pgfusepath{stroke}%
\end{pgfscope}%
\begin{pgfscope}%
\pgfsetbuttcap%
\pgfsetroundjoin%
\definecolor{currentfill}{rgb}{0.000000,0.000000,0.000000}%
\pgfsetfillcolor{currentfill}%
\pgfsetlinewidth{0.803000pt}%
\definecolor{currentstroke}{rgb}{0.000000,0.000000,0.000000}%
\pgfsetstrokecolor{currentstroke}%
\pgfsetdash{}{0pt}%
\pgfsys@defobject{currentmarker}{\pgfqpoint{-0.048611in}{0.000000in}}{\pgfqpoint{-0.000000in}{0.000000in}}{%
\pgfpathmoveto{\pgfqpoint{-0.000000in}{0.000000in}}%
\pgfpathlineto{\pgfqpoint{-0.048611in}{0.000000in}}%
\pgfusepath{stroke,fill}%
}%
\begin{pgfscope}%
\pgfsys@transformshift{0.953437in}{4.973398in}%
\pgfsys@useobject{currentmarker}{}%
\end{pgfscope}%
\end{pgfscope}%
\begin{pgfscope}%
\definecolor{textcolor}{rgb}{0.000000,0.000000,0.000000}%
\pgfsetstrokecolor{textcolor}%
\pgfsetfillcolor{textcolor}%
\pgftext[x=0.546970in, y=4.920637in, left, base]{\color{textcolor}{\rmfamily\fontsize{10.000000}{12.000000}\selectfont\catcode`\^=\active\def^{\ifmmode\sp\else\^{}\fi}\catcode`\%=\active\def%{\%}0.75}}%
\end{pgfscope}%
\begin{pgfscope}%
\definecolor{textcolor}{rgb}{0.000000,0.000000,0.000000}%
\pgfsetstrokecolor{textcolor}%
\pgfsetfillcolor{textcolor}%
\pgftext[x=0.383390in,y=3.029063in,,bottom,rotate=90.000000]{\color{textcolor}{\rmfamily\fontsize{11.000000}{13.200000}\selectfont\catcode`\^=\active\def^{\ifmmode\sp\else\^{}\fi}\catcode`\%=\active\def%{\%}$\beta$ (U-Beta)}}%
\end{pgfscope}%
\begin{pgfscope}%
\pgfpathrectangle{\pgfqpoint{0.953437in}{0.609444in}}{\pgfqpoint{4.839236in}{4.839236in}}%
\pgfusepath{clip}%
\pgfsetrectcap%
\pgfsetroundjoin%
\pgfsetlinewidth{1.204500pt}%
\definecolor{currentstroke}{rgb}{0.200000,0.200000,0.200000}%
\pgfsetstrokecolor{currentstroke}%
\pgfsetdash{}{0pt}%
\pgfpathmoveto{\pgfqpoint{5.101354in}{3.029063in}}%
\pgfpathlineto{\pgfqpoint{4.237205in}{4.525813in}}%
\pgfpathlineto{\pgfqpoint{2.508906in}{4.525813in}}%
\pgfpathlineto{\pgfqpoint{1.644757in}{3.029063in}}%
\pgfpathlineto{\pgfqpoint{2.508906in}{1.532312in}}%
\pgfpathlineto{\pgfqpoint{4.237205in}{1.532312in}}%
\pgfpathlineto{\pgfqpoint{5.101354in}{3.029063in}}%
\pgfusepath{stroke}%
\end{pgfscope}%
\begin{pgfscope}%
\pgfpathrectangle{\pgfqpoint{0.953437in}{0.609444in}}{\pgfqpoint{4.839236in}{4.839236in}}%
\pgfusepath{clip}%
\pgfsetrectcap%
\pgfsetroundjoin%
\pgfsetlinewidth{0.501875pt}%
\definecolor{currentstroke}{rgb}{0.000000,0.000000,0.000000}%
\pgfsetstrokecolor{currentstroke}%
\pgfsetdash{}{0pt}%
\pgfpathmoveto{\pgfqpoint{0.953437in}{3.029063in}}%
\pgfpathlineto{\pgfqpoint{5.792674in}{3.029063in}}%
\pgfusepath{stroke}%
\end{pgfscope}%
\begin{pgfscope}%
\pgfpathrectangle{\pgfqpoint{0.953437in}{0.609444in}}{\pgfqpoint{4.839236in}{4.839236in}}%
\pgfusepath{clip}%
\pgfsetrectcap%
\pgfsetroundjoin%
\pgfsetlinewidth{0.501875pt}%
\definecolor{currentstroke}{rgb}{0.000000,0.000000,0.000000}%
\pgfsetstrokecolor{currentstroke}%
\pgfsetdash{}{0pt}%
\pgfpathmoveto{\pgfqpoint{3.373056in}{0.609444in}}%
\pgfpathlineto{\pgfqpoint{3.373056in}{5.448681in}}%
\pgfusepath{stroke}%
\end{pgfscope}%
\begin{pgfscope}%
\pgfsetrectcap%
\pgfsetmiterjoin%
\pgfsetlinewidth{0.803000pt}%
\definecolor{currentstroke}{rgb}{0.000000,0.000000,0.000000}%
\pgfsetstrokecolor{currentstroke}%
\pgfsetdash{}{0pt}%
\pgfpathmoveto{\pgfqpoint{0.953437in}{0.609444in}}%
\pgfpathlineto{\pgfqpoint{0.953437in}{5.448681in}}%
\pgfusepath{stroke}%
\end{pgfscope}%
\begin{pgfscope}%
\pgfsetrectcap%
\pgfsetmiterjoin%
\pgfsetlinewidth{0.803000pt}%
\definecolor{currentstroke}{rgb}{0.000000,0.000000,0.000000}%
\pgfsetstrokecolor{currentstroke}%
\pgfsetdash{}{0pt}%
\pgfpathmoveto{\pgfqpoint{5.792674in}{0.609444in}}%
\pgfpathlineto{\pgfqpoint{5.792674in}{5.448681in}}%
\pgfusepath{stroke}%
\end{pgfscope}%
\begin{pgfscope}%
\pgfsetrectcap%
\pgfsetmiterjoin%
\pgfsetlinewidth{0.803000pt}%
\definecolor{currentstroke}{rgb}{0.000000,0.000000,0.000000}%
\pgfsetstrokecolor{currentstroke}%
\pgfsetdash{}{0pt}%
\pgfpathmoveto{\pgfqpoint{0.953437in}{0.609444in}}%
\pgfpathlineto{\pgfqpoint{5.792674in}{0.609444in}}%
\pgfusepath{stroke}%
\end{pgfscope}%
\begin{pgfscope}%
\pgfsetrectcap%
\pgfsetmiterjoin%
\pgfsetlinewidth{0.803000pt}%
\definecolor{currentstroke}{rgb}{0.000000,0.000000,0.000000}%
\pgfsetstrokecolor{currentstroke}%
\pgfsetdash{}{0pt}%
\pgfpathmoveto{\pgfqpoint{0.953437in}{5.448681in}}%
\pgfpathlineto{\pgfqpoint{5.792674in}{5.448681in}}%
\pgfusepath{stroke}%
\end{pgfscope}%
\begin{pgfscope}%
\pgfsetroundcap%
\pgfsetroundjoin%
\pgfsetlinewidth{1.003750pt}%
\definecolor{currentstroke}{rgb}{1.000000,0.000000,0.000000}%
\pgfsetstrokecolor{currentstroke}%
\pgfsetdash{}{0pt}%
\pgfpathmoveto{\pgfqpoint{3.400825in}{3.029063in}}%
\pgfpathquadraticcurveto{\pgfqpoint{4.237194in}{3.029063in}}{\pgfqpoint{5.058036in}{3.029063in}}%
\pgfusepath{stroke}%
\end{pgfscope}%
\begin{pgfscope}%
\pgfsetroundcap%
\pgfsetroundjoin%
\pgfsetlinewidth{1.003750pt}%
\definecolor{currentstroke}{rgb}{1.000000,0.000000,0.000000}%
\pgfsetstrokecolor{currentstroke}%
\pgfsetdash{}{0pt}%
\pgfpathmoveto{\pgfqpoint{4.974702in}{3.070729in}}%
\pgfpathlineto{\pgfqpoint{5.058036in}{3.029063in}}%
\pgfpathlineto{\pgfqpoint{4.974702in}{2.987396in}}%
\pgfusepath{stroke}%
\end{pgfscope}%
\begin{pgfscope}%
\definecolor{textcolor}{rgb}{0.000000,0.000000,0.000000}%
\pgfsetstrokecolor{textcolor}%
\pgfsetfillcolor{textcolor}%
\pgftext[x=5.447014in,y=3.029063in,,]{\color{textcolor}{\rmfamily\fontsize{12.000000}{14.400000}\selectfont\catcode`\^=\active\def^{\ifmmode\sp\else\^{}\fi}\catcode`\%=\active\def%{\%}$V_{1}$}}%
\end{pgfscope}%
\begin{pgfscope}%
\pgfsetroundcap%
\pgfsetroundjoin%
\pgfsetlinewidth{1.003750pt}%
\definecolor{currentstroke}{rgb}{1.000000,0.000000,0.000000}%
\pgfsetstrokecolor{currentstroke}%
\pgfsetdash{}{0pt}%
\pgfpathmoveto{\pgfqpoint{3.386940in}{3.053112in}}%
\pgfpathquadraticcurveto{\pgfqpoint{3.805125in}{3.777429in}}{\pgfqpoint{4.215546in}{4.488298in}}%
\pgfusepath{stroke}%
\end{pgfscope}%
\begin{pgfscope}%
\pgfsetroundcap%
\pgfsetroundjoin%
\pgfsetlinewidth{1.003750pt}%
\definecolor{currentstroke}{rgb}{1.000000,0.000000,0.000000}%
\pgfsetstrokecolor{currentstroke}%
\pgfsetdash{}{0pt}%
\pgfpathmoveto{\pgfqpoint{4.137795in}{4.436963in}}%
\pgfpathlineto{\pgfqpoint{4.215546in}{4.488298in}}%
\pgfpathlineto{\pgfqpoint{4.209963in}{4.395296in}}%
\pgfusepath{stroke}%
\end{pgfscope}%
\begin{pgfscope}%
\definecolor{textcolor}{rgb}{0.000000,0.000000,0.000000}%
\pgfsetstrokecolor{textcolor}%
\pgfsetfillcolor{textcolor}%
\pgftext[x=4.410035in,y=4.825163in,,]{\color{textcolor}{\rmfamily\fontsize{12.000000}{14.400000}\selectfont\catcode`\^=\active\def^{\ifmmode\sp\else\^{}\fi}\catcode`\%=\active\def%{\%}$V_{2}$}}%
\end{pgfscope}%
\begin{pgfscope}%
\pgfsetroundcap%
\pgfsetroundjoin%
\pgfsetlinewidth{1.003750pt}%
\definecolor{currentstroke}{rgb}{1.000000,0.000000,0.000000}%
\pgfsetstrokecolor{currentstroke}%
\pgfsetdash{}{0pt}%
\pgfpathmoveto{\pgfqpoint{3.359171in}{3.053112in}}%
\pgfpathquadraticcurveto{\pgfqpoint{2.940986in}{3.777429in}}{\pgfqpoint{2.530566in}{4.488298in}}%
\pgfusepath{stroke}%
\end{pgfscope}%
\begin{pgfscope}%
\pgfsetroundcap%
\pgfsetroundjoin%
\pgfsetlinewidth{1.003750pt}%
\definecolor{currentstroke}{rgb}{1.000000,0.000000,0.000000}%
\pgfsetstrokecolor{currentstroke}%
\pgfsetdash{}{0pt}%
\pgfpathmoveto{\pgfqpoint{2.536148in}{4.395296in}}%
\pgfpathlineto{\pgfqpoint{2.530566in}{4.488298in}}%
\pgfpathlineto{\pgfqpoint{2.608317in}{4.436963in}}%
\pgfusepath{stroke}%
\end{pgfscope}%
\begin{pgfscope}%
\definecolor{textcolor}{rgb}{0.000000,0.000000,0.000000}%
\pgfsetstrokecolor{textcolor}%
\pgfsetfillcolor{textcolor}%
\pgftext[x=2.336076in,y=4.825163in,,]{\color{textcolor}{\rmfamily\fontsize{12.000000}{14.400000}\selectfont\catcode`\^=\active\def^{\ifmmode\sp\else\^{}\fi}\catcode`\%=\active\def%{\%}$V_{3}$}}%
\end{pgfscope}%
\begin{pgfscope}%
\pgfsetroundcap%
\pgfsetroundjoin%
\pgfsetlinewidth{1.003750pt}%
\definecolor{currentstroke}{rgb}{1.000000,0.000000,0.000000}%
\pgfsetstrokecolor{currentstroke}%
\pgfsetdash{}{0pt}%
\pgfpathmoveto{\pgfqpoint{3.345286in}{3.029063in}}%
\pgfpathquadraticcurveto{\pgfqpoint{2.508917in}{3.029063in}}{\pgfqpoint{1.688076in}{3.029063in}}%
\pgfusepath{stroke}%
\end{pgfscope}%
\begin{pgfscope}%
\pgfsetroundcap%
\pgfsetroundjoin%
\pgfsetlinewidth{1.003750pt}%
\definecolor{currentstroke}{rgb}{1.000000,0.000000,0.000000}%
\pgfsetstrokecolor{currentstroke}%
\pgfsetdash{}{0pt}%
\pgfpathmoveto{\pgfqpoint{1.771409in}{2.987396in}}%
\pgfpathlineto{\pgfqpoint{1.688076in}{3.029063in}}%
\pgfpathlineto{\pgfqpoint{1.771409in}{3.070729in}}%
\pgfusepath{stroke}%
\end{pgfscope}%
\begin{pgfscope}%
\definecolor{textcolor}{rgb}{0.000000,0.000000,0.000000}%
\pgfsetstrokecolor{textcolor}%
\pgfsetfillcolor{textcolor}%
\pgftext[x=1.299097in,y=3.029063in,,]{\color{textcolor}{\rmfamily\fontsize{12.000000}{14.400000}\selectfont\catcode`\^=\active\def^{\ifmmode\sp\else\^{}\fi}\catcode`\%=\active\def%{\%}$V_{4}$}}%
\end{pgfscope}%
\begin{pgfscope}%
\pgfsetroundcap%
\pgfsetroundjoin%
\pgfsetlinewidth{1.003750pt}%
\definecolor{currentstroke}{rgb}{1.000000,0.000000,0.000000}%
\pgfsetstrokecolor{currentstroke}%
\pgfsetdash{}{0pt}%
\pgfpathmoveto{\pgfqpoint{3.359171in}{3.005013in}}%
\pgfpathquadraticcurveto{\pgfqpoint{2.940986in}{2.280696in}}{\pgfqpoint{2.530566in}{1.569827in}}%
\pgfusepath{stroke}%
\end{pgfscope}%
\begin{pgfscope}%
\pgfsetroundcap%
\pgfsetroundjoin%
\pgfsetlinewidth{1.003750pt}%
\definecolor{currentstroke}{rgb}{1.000000,0.000000,0.000000}%
\pgfsetstrokecolor{currentstroke}%
\pgfsetdash{}{0pt}%
\pgfpathmoveto{\pgfqpoint{2.608317in}{1.621162in}}%
\pgfpathlineto{\pgfqpoint{2.530566in}{1.569827in}}%
\pgfpathlineto{\pgfqpoint{2.536148in}{1.662829in}}%
\pgfusepath{stroke}%
\end{pgfscope}%
\begin{pgfscope}%
\definecolor{textcolor}{rgb}{0.000000,0.000000,0.000000}%
\pgfsetstrokecolor{textcolor}%
\pgfsetfillcolor{textcolor}%
\pgftext[x=2.336076in,y=1.232962in,,]{\color{textcolor}{\rmfamily\fontsize{12.000000}{14.400000}\selectfont\catcode`\^=\active\def^{\ifmmode\sp\else\^{}\fi}\catcode`\%=\active\def%{\%}$V_{5}$}}%
\end{pgfscope}%
\begin{pgfscope}%
\pgfsetroundcap%
\pgfsetroundjoin%
\pgfsetlinewidth{1.003750pt}%
\definecolor{currentstroke}{rgb}{1.000000,0.000000,0.000000}%
\pgfsetstrokecolor{currentstroke}%
\pgfsetdash{}{0pt}%
\pgfpathmoveto{\pgfqpoint{3.386940in}{3.005013in}}%
\pgfpathquadraticcurveto{\pgfqpoint{3.805125in}{2.280696in}}{\pgfqpoint{4.215546in}{1.569827in}}%
\pgfusepath{stroke}%
\end{pgfscope}%
\begin{pgfscope}%
\pgfsetroundcap%
\pgfsetroundjoin%
\pgfsetlinewidth{1.003750pt}%
\definecolor{currentstroke}{rgb}{1.000000,0.000000,0.000000}%
\pgfsetstrokecolor{currentstroke}%
\pgfsetdash{}{0pt}%
\pgfpathmoveto{\pgfqpoint{4.209963in}{1.662829in}}%
\pgfpathlineto{\pgfqpoint{4.215546in}{1.569827in}}%
\pgfpathlineto{\pgfqpoint{4.137795in}{1.621162in}}%
\pgfusepath{stroke}%
\end{pgfscope}%
\begin{pgfscope}%
\definecolor{textcolor}{rgb}{0.000000,0.000000,0.000000}%
\pgfsetstrokecolor{textcolor}%
\pgfsetfillcolor{textcolor}%
\pgftext[x=4.410035in,y=1.232962in,,]{\color{textcolor}{\rmfamily\fontsize{12.000000}{14.400000}\selectfont\catcode`\^=\active\def^{\ifmmode\sp\else\^{}\fi}\catcode`\%=\active\def%{\%}$V_{6}$}}%
\end{pgfscope}%
\begin{pgfscope}%
\definecolor{textcolor}{rgb}{0.501961,0.501961,0.501961}%
\pgfsetstrokecolor{textcolor}%
\pgfsetfillcolor{textcolor}%
\pgftext[x=4.271106in,y=3.547552in,left,base]{\color{textcolor}{\rmfamily\fontsize{10.000000}{12.000000}\bfseries\selectfont\catcode`\^=\active\def^{\ifmmode\sp\else\^{}\fi}\catcode`\%=\active\def%{\%}S1}}%
\end{pgfscope}%
\begin{pgfscope}%
\definecolor{textcolor}{rgb}{0.501961,0.501961,0.501961}%
\pgfsetstrokecolor{textcolor}%
\pgfsetfillcolor{textcolor}%
\pgftext[x=3.373056in,y=4.066042in,left,base]{\color{textcolor}{\rmfamily\fontsize{10.000000}{12.000000}\bfseries\selectfont\catcode`\^=\active\def^{\ifmmode\sp\else\^{}\fi}\catcode`\%=\active\def%{\%}S2}}%
\end{pgfscope}%
\begin{pgfscope}%
\definecolor{textcolor}{rgb}{0.501961,0.501961,0.501961}%
\pgfsetstrokecolor{textcolor}%
\pgfsetfillcolor{textcolor}%
\pgftext[x=2.475005in,y=3.547552in,left,base]{\color{textcolor}{\rmfamily\fontsize{10.000000}{12.000000}\bfseries\selectfont\catcode`\^=\active\def^{\ifmmode\sp\else\^{}\fi}\catcode`\%=\active\def%{\%}S3}}%
\end{pgfscope}%
\begin{pgfscope}%
\definecolor{textcolor}{rgb}{0.501961,0.501961,0.501961}%
\pgfsetstrokecolor{textcolor}%
\pgfsetfillcolor{textcolor}%
\pgftext[x=2.475005in,y=2.510573in,left,base]{\color{textcolor}{\rmfamily\fontsize{10.000000}{12.000000}\bfseries\selectfont\catcode`\^=\active\def^{\ifmmode\sp\else\^{}\fi}\catcode`\%=\active\def%{\%}S4}}%
\end{pgfscope}%
\begin{pgfscope}%
\definecolor{textcolor}{rgb}{0.501961,0.501961,0.501961}%
\pgfsetstrokecolor{textcolor}%
\pgfsetfillcolor{textcolor}%
\pgftext[x=3.373056in,y=1.992083in,left,base]{\color{textcolor}{\rmfamily\fontsize{10.000000}{12.000000}\bfseries\selectfont\catcode`\^=\active\def^{\ifmmode\sp\else\^{}\fi}\catcode`\%=\active\def%{\%}S5}}%
\end{pgfscope}%
\begin{pgfscope}%
\definecolor{textcolor}{rgb}{0.501961,0.501961,0.501961}%
\pgfsetstrokecolor{textcolor}%
\pgfsetfillcolor{textcolor}%
\pgftext[x=4.271106in,y=2.510573in,left,base]{\color{textcolor}{\rmfamily\fontsize{10.000000}{12.000000}\bfseries\selectfont\catcode`\^=\active\def^{\ifmmode\sp\else\^{}\fi}\catcode`\%=\active\def%{\%}S6}}%
\end{pgfscope}%
\begin{pgfscope}%
\definecolor{textcolor}{rgb}{0.000000,0.000000,0.000000}%
\pgfsetstrokecolor{textcolor}%
\pgfsetfillcolor{textcolor}%
\pgftext[x=3.373056in,y=5.726458in,,base]{\color{textcolor}{\rmfamily\fontsize{12.000000}{14.400000}\selectfont\catcode`\^=\active\def^{\ifmmode\sp\else\^{}\fi}\catcode`\%=\active\def%{\%}Raumzeiger-Hexagon (SVPWM)}}%
\end{pgfscope}%
\end{pgfpicture}%
\makeatother%
\endgroup%

    \caption{PGF-Grafik aus Matplotlib}
\end{figure}