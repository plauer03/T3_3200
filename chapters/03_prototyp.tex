\chapter{Voruntersuchung und Prototyping}
\label{chap:prototyp}

\section{Teilespender Hoverboard}
\todo[inline]{Text: Motivation Recycling. Entnahme K4145 MOSFETs und FDS2103S Treiber.}
\todo[inline, color=blue!40]{BILD: Foto der Hoverboard-Platine oder der ausgelöteten Teile.}

% SMD lötaufsatz:
\subsection*{SMD Lötaufsatz}
\begin{tikzpicture}
	% Paths, nodes and wires:
	\node[shape=rectangle, draw, line width=1pt, minimum width=3.965cm, minimum height=1.965cm] at (6, 5){};
	\draw (4.5, 6) -- (4.5, 7);
	\draw (5.5, 6) -- (5.5, 7);
	\draw (6.5, 6) -- (6.5, 7);
	\draw (7.5, 6) -- (7.5, 7);
	\draw (4.5, 3) -- (4.5, 4);
	\draw (5.5, 3) -- (5.5, 4);
	\draw (6.5, 3) -- (6.5, 4);
	\draw (7.5, 3) -- (7.5, 4);
	\node[shape=rectangle, minimum width=0.965cm, minimum height=0.965cm] at (4.5, 5.5){} node[anchor=center, align=center, text width=0.577cm, inner sep=6pt] at (4.5, 5.5){GND};
	\node[shape=rectangle, minimum width=0.965cm, minimum height=0.965cm] at (5.5, 5.5){} node[anchor=center, align=center, text width=0.577cm, inner sep=6pt] at (5.5, 5.5){LIN*};
	\node[shape=rectangle, minimum width=0.965cm, minimum height=0.965cm] at (6.5, 5.5){} node[anchor=center, align=center, text width=0.577cm, inner sep=6pt] at (6.5, 5.5){HIN};
	\node[shape=rectangle, minimum width=0.965cm, minimum height=0.965cm] at (7.5, 5.5){} node[anchor=center, align=center, text width=0.577cm, inner sep=6pt] at (7.5, 5.5){VCC};
	\node[shape=rectangle, minimum width=0.965cm, minimum height=0.965cm] at (4.5, 4.5){} node[anchor=center, align=center, text width=0.577cm, inner sep=6pt] at (4.5, 4.5){LO};
	\node[shape=rectangle, minimum width=0.965cm, minimum height=0.965cm] at (5.5, 4.5){} node[anchor=center, align=center, text width=0.577cm, inner sep=6pt] at (5.5, 4.5){VS};
	\node[shape=rectangle, minimum width=0.965cm, minimum height=0.965cm] at (6.5, 4.5){} node[anchor=center, align=center, text width=0.577cm, inner sep=6pt] at (6.5, 4.5){HO};
	\node[shape=rectangle, minimum width=0.965cm, minimum height=0.965cm] at (7.5, 4.5){} node[anchor=center, align=center, text width=0.577cm, inner sep=6pt] at (7.5, 4.5){VB};
	\node[shape=circle, draw, line width=1pt, minimum width=0.465cm] at (3.677, 6.323){};
	\draw (9, 6) to[empty diode] (9, 4);
	\draw (9, 6) -| (9, 7) -- (7.5, 7);
	\draw (7.5, 3) -- (9, 3) -| (9, 4);
	\node[sground, xscale=-1, yscale=-1] at (4.5, 7){};
	\draw (7.5, 2) to[capacitor, l={$220\space\mathrm{nF} /63\space\mathrm{V}$}] (5.5, 2);
	\draw (5.5, 2) -| (5.5, 3);
	\draw (7.5, 2) -| (7.5, 3);
\end{tikzpicture}

\section{Diskreter Aufbau (Breadboard)}
\todo[inline]{Text: Aufbau der Testschaltung auf dem Steckbrett. Herausforderung: Induktivitäten.}
\todo[inline, color=blue!40]{BILD: Foto des Breadboard-Aufbaus (zeigt 'Bastel'-Charakter).}

\section{Erste Inbetriebnahme}
\todo[inline]{Text: Ansteuerung mit Arduino DUE (3.3V Logic). Erste Drehversuche (Open Loop).}
\todo[inline, color=blue!40]{VERWEIS: Der Test-Code für den Arduino befindet sich in Anhang C.}