\chapter{Entwurf der Leistungselektronik}
\label{chap:entwurf_leistung}

\section{MOSFET-Auswahl}
\todo[inline]{Hier wird die Auswahl der MOSFETs begründet: Sperrspannung, R$_{DS(on)}$, Gateladung, Gehäusewahl usw.}

\section{Verlustanalyse}
\todo[inline]{Berechnung und Abschätzung der statischen und dynamischen Verluste pro MOSFET/Phase.}

\section{Gate-Treiber}
\todo[inline]{Beschreibung des gewählten Treibers, Totzeiten, Schutzfunktionen, Ansteuerlogik.}

\section{Bootstrap}
\label{sec:bootstrap}

Für die Ansteuerung der High-Side-MOSFETs in der Halbbrücke wird ein Bootstrap-Konzept verwendet, bei dem ein Kondensator als kurzzeitige, schwimmende Spannungsquelle dient. 
Dieser \emph{Bootstrap-Kondensator} wird während der Leitphase des Low-Side-MOSFETs aufgeladen und liefert die notwendige Energie, um das Gate des High-Side-MOSFETs vollständig zu treiben. 
Ohne eine korrekt dimensionierte Bootstrap-Schaltung kann die Gate-Spannung während der Einschaltzeit des High-Side-MOSFETs unter den erforderlichen Wert fallen, was zu erhöhten Leitverlusten oder einem Auslösen der Unterspannungsabschaltung des Treibers führen würde.

Die Dimensionierung des Bootstrap-Kondensators basiert auf der Gesamtladung, die während einer Einschaltphase bereitgestellt werden muss. Diese setzt sich aus der Gateladung des MOSFETs, dem Energiebedarf des Level-Shifters im Treiber sowie den Verlusten durch Leckströme zusammen. 
Für die vorliegende Schaltung mit dem MOSFET \textbf{IPP034N08N5} und dem Treiber \textbf{IR2104} wurde die Berechnung auf Basis der MOSFET-Datenblätter sowie der Applikationsanleitung AN-6076 von ON Semiconductor durchgeführt \cite{AN6076}. 
Hierbei wurden alle relevanten Parameter wie Gateladung, Eingangskapazität, Leckströme des Treibers und der Bootstrap-Diode sowie die maximale Einschaltdauer der High-Side-Schalter berücksichtigt, um eine robuste Auslegung auch im Worst-Case zu gewährleisten.

In Tabelle \ref{tab:bootstrap_params} sind die wesentlichen Größen zusammengefasst, die in die Berechnung der erforderlichen Kapazität eingeflossen sind:

\begin{table}[h!]
    \centering
    \caption{Parameter zur Dimensionierung des Bootstrap-Kondensators}
    \label{tab:bootstrap_params}
    \begin{tabular}{l c c}
        \toprule
        Größe & Symbol & Wert \\
        \midrule
        Gateladung MOSFET & $Q_\mathrm{Gate}$ & \SI{111}{\nano\coulomb} \\
        Level-Shifter-Verbrauch & $Q_\mathrm{LS}$ & \SI{3}{\nano\coulomb} \\
        Leckströme während $t_\mathrm{on}$ & $Q_\mathrm{Leak}$ & \SI{5,8}{\nano\coulomb} \\
        Gesamtladung & $Q_\mathrm{Total}$ & \SI{120}{\nano\coulomb} \\
        Maximaler Spannungsabfall & $\Delta V_\mathrm{BOOT}$ & \SI{1}{\volt} \\
        Minimale Kapazität & $C_\mathrm{BOOT,min}$ & \SI{120}{\nano\farad} \\
        \bottomrule
    \end{tabular}
\end{table}

Auf Basis dieser Analyse wurde für die praktische Umsetzung ein Keramikkondensator mit \SI{220}{\nano\farad} gewählt, um Fertigungstoleranzen, Alterungseffekte sowie den DC-Bias-Einfluss bei Keramikkondensatoren ausreichend zu berücksichtigen. 
Die hier dargestellte Vorgehensweise zeigt, dass die Auslegung des Bootstrap-Kondensators nicht trivial ist und eine detaillierte Analyse unter Berücksichtigung der Datenblattangaben und der einschlägigen Applikationsliteratur notwendig ist, um sowohl Zuverlässigkeit als auch Effizienz der Schaltung sicherzustellen. 
Eine vollständige, tiefgehende Berechnung inklusive aller Zwischenwerte und Datenblattangaben ist im Anhang \ref{app:bootstrap} dokumentiert.


\section{Zwischenkreis}
\todo[inline]{Dimensionierung des DC-Links: Kapazität, Ripple, Wahl der Elkos und Spannungsreserve.}
