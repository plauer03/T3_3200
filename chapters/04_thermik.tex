\chapter{Thermisches Management}
\label{chap:thermik}
\todo[inline, color=green!40]{Ingenieurs-Aspekt: Kühlkörperberechnung.}

\section{Thermisches Ersatzschaltbild}
Um die Junction-Temperatur $T_J$ unter dem Maximum (\SI{175}{\celsius}) zu halten, gilt:
\begin{equation}
    T_J = P_{tot} \cdot (R_{thJC} + R_{thCH} + R_{thHA}) + T_{Amb}
\end{equation}
Dabei ist:
\begin{itemize}
    \item $R_{thJC}$: Wärmewiderstand Chip-Gehäuse (aus Datenblatt: \SI{1,1}{\kelvin\per\watt}).
    \item $R_{thCH}$: Wärmewiderstand Gehäuse-Kühlkörper (Isolierscheibe! Glimmer $\approx \SI{0,5}{\kelvin\per\watt}$).
    \item $R_{thHA}$: Wärmewiderstand Kühlkörper-Luft (zu berechnen).
\end{itemize}

\section{Auslegung des Kühlkörpers}
\todo[inline]{Rechnung: Welchen Rth-Wert muss der Kühlkörper haben, um bei 30A (ca. 4-5W Gesamtverlust pro FET) stabil zu bleiben?}