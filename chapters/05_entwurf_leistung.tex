\chapter{Entwurf der Leistungselektronik}
\label{chap:entwurf}

\section{Anforderungsanalyse}
\todo[inline]{Text: 30A Dauerstrom, 42V Spitzen, passiv gekühlt.}

\section{Auslegung der B6-Endstufe}
\todo[inline]{Text: MOSFET-Wahl (Infineon), High-Side/Low-Side Prinzip.}
\todo[inline, color=blue!40]{BILD: Prinzipschaltbild EINER Halbbrücke (mit Treiber, R\_G, Diode). Nicht der ganze Plan!}

\section{Gate-Beschaltung}
\todo[inline]{Text: Dimensionierung Gate-Vorwiderstand (z.B. 10 Ohm) und Pulldown (10k).}

\section{Bootstrap-Schaltung}
\todo[inline]{Text: Berechnung C\_Boot für den IR2104. Formeln aus unseren Notizen (Q\_total / dV).}
\todo[inline, color=blue!40]{BILD: Skizze der Bootstrap-Schleife.}

\section{Verlustleistung und Thermik}
\todo[inline]{Text: Berechnung P\_Cond + P\_SW.}
\todo[inline]{Text: Berechnung R\_th (Wärmewiderstand) für den Kühlkörper.}
\todo[inline, color=blue!40]{BILD: Thermisches Ersatzschaltbild.}
Folie 133 Leistungselektronik Gate Vorwiderstand berechnen mit $t_{rr}$