\chapter{Aufbau und Messungen}
\label{chap:aufbau_messungen}

\section{Hardwareaufbau}
\todo[inline]{Text: Aufbau des Prototyps beschreiben (THT-Aufbau oder Lochraster).}
\todo[inline]{Text: Stromführung, Leiterquerschnitte und Bauteilplatzierung erläutern.}
\todo[inline]{Text: Besonderheiten im Layout, z.B. Kühlkörper, EMV-relevante Punkte.}
\todo[inline]{Optional: Foto oder Schema des Prototyps einfügen.}

\section{Inbetriebnahme}
\todo[inline]{Text: Erste Tests ohne Last (Smoke-Test, Funktionskontrolle).}
\todo[inline]{Text: Beschreibung des Testaufbaus (Versorgungsspannung, Messpunkte, Signalzugänge).}
\todo[inline]{Text: Sicherheitsmaßnahmen und Schutzschaltungen erläutern.}

\section{Messungen}
\todo[inline]{Text: Temperaturmessungen beschreiben (Thermofühler, IR-Kamera, Messpunkte).}
\todo[inline]{Text: Elektrische Messungen (Strom, Spannung, ggf. PWM-Signale) dokumentieren.}
\todo[inline]{Text: Messbedingungen und Lastprofile angeben.}
\todo[inline]{Optional: Diagramme der gemessenen Größen einfügen.}

\section{Vergleich Rechnung und Messung}
\todo[inline]{Tabelle: berechnete vs. gemessene Temperaturen / Verlustleistungen erstellen.}
\todo[inline]{Text: Abweichungsanalyse (mögliche Ursachen für Unterschiede).}

\section{Bewertung des thermischen Konzepts}
\todo[inline]{Text: Grenztemperaturen eingehalten?}
\todo[inline]{Text: Bewertung Kühlkörper / Wärmeabfuhr.}
\todo[inline]{Text: Verbesserungspotential aufzeigen.}
