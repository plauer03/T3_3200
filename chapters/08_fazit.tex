\chapter{Fazit und Ausblick}
\label{chap:fazit}

\section{Zusammenfassung}
In dieser Studienarbeit wurde erfolgreich ein Motorcontroller für einen \SI{36}{\volt} E-Kart-Antrieb entwickelt. Die Hardware wurde für Ströme bis \SI{30}{\ampere} dimensioniert, wobei besonderer Wert auf die thermische Auslegung und die Dimensionierung der Bootstrap-Schaltung gelegt wurde.

Die Entscheidung für die Ansteuerung mittels SVPWM (statt FOC) hat sich für diesen Anwendungsfall als zielführend erwiesen. Sie ermöglichte eine schnelle Inbetriebnahme und einen zuverlässigen Motoranlauf, ohne die hohe Rechenlast und Reglerkomplexität einer FOC-Lösung.

\section{Ausblick}
Für zukünftige Iterationen des Projekts könnten folgende Erweiterungen betrachtet werden:
\begin{itemize}
    \item Integration einer Strommessung zur Überlastsicherung (Hardware-Schutz).
    \item Untersuchung, ob eine FOC-Regelung bei niedrigen Drehzahlen akustische Vorteile bietet (obwohl die SVPWM bereits effizient arbeitet).
    \item Layout-Optimierung zur weiteren Verkleinerung der Platine.
\end{itemize}