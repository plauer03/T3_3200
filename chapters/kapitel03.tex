\section{Dimensionierung des Bootstrap-Kondensators $C_\mathrm{BOOT}$}

Die Dimensionierung des Bootstrap-Kondensators ($C_\mathrm{BOOT}$) ist ein kritischer Aspekt beim Design von High-Side-Treiberschaltungen. Der Kondensator dient als schwimmende Spannungsquelle (Floating Supply) für den High-Side-Teil des Gatetreibers und wird jeweils aufgeladen, wenn der Low-Side-Schalter leitend ist und das Referenzpotential auf Masse zieht. Entladen wird der Kondensator ausschließlich während der Einschaltphase des High-Side-MOSFETs.

Ein wesentliches Auslegungskriterium ist der maximal zulässige Spannungsabfall $\Delta V_\mathrm{BOOT}$ über dem Kondensator während der Einschaltzeit ($t_\mathrm{on}$). Fällt die Spannung am Kondensator zu stark ab, sinkt die effektive Gate-Source-Spannung $V_\mathrm{GS}$ des High-Side-Transistors. Dies kann dazu führen, dass der Transistor den Sättigungsbereich verlässt, was die Leitendverluste signifikant erhöht, oder dass die Unterspannungsabschaltung (UVLO) des Treibers aktiviert wird. Gemäß der Auslegungsvorschriften für Bootstrap-Schaltungen muss der Kondensator daher eine ausreichende Ladungsreserve $Q_\mathrm{Total}$ bereitstellen, um den Spannungsabfall innerhalb definierter Grenzen zu halten \cite{Quelle_AN6076}.

Die zur Dimensionierung notwendige Gesamtkapazität berechnet sich nach:
\begin{equation}
C_\mathrm{BOOT} = \frac{Q_\mathrm{Total}}{\Delta V_\mathrm{BOOT}}
\label{eq:cboot_def}
\end{equation}
Die benötigte Gesamtladung $Q_\mathrm{Total}$ setzt sich dabei additiv aus der Gateladung des MOSFETs, der Ladung für den internen Level-Shifter sowie den Verlusten durch Leckströme zusammen.

Für den hier verwendeten MOSFET IPP034N08N5 ist im Datenblatt eine Gateladung von \SI{87}{\nano\coulomb} bei einer Ansteuerung von \SI{10}{\volt} spezifiziert. Da der Gatetreiber IR2104 jedoch mit $V_\mathrm{CC} = \SI{15}{\volt}$ versorgt wird, muss die Gateladung für diesen Arbeitspunkt extrapoliert werden. Unter Einbeziehung der Eingangskapazität $C_\mathrm{iss}$ ergibt sich eine effektive Gateladung von:
\begin{equation}
Q_\mathrm{Gate} \approx \SI{5}{\volt} \cdot C_\mathrm{iss} + Q_\mathrm{Gate,10V} = \SI{111}{\nano\coulomb}
\end{equation}

Zusätzlich zur statischen Gateladung müssen die zeitabhängigen Verluste durch Leckströme während der maximalen Einschaltdauer berücksichtigt werden. Für die Berechnung wird ein Worst-Case-Szenario mit einer Periodendauer $T = \SI{50}{\micro\second}$ (entsprechend $f_s = \SI{20}{\kilo\hertz}$) als relevante Entladezeit $t_\mathrm{on}$ angenommen. Die relevanten Leckströme ergeben sich aus den Datenblättern des Treibers und der Peripherie:
\begin{itemize}
    \item Treiber-Leckstrom: $I_\mathrm{LK} = \SI{50}{\micro\ampere}$
    \item Ruhestrom der Bootstrap-Schaltung: $I_\mathrm{QBS} = \SI{55}{\micro\ampere}$
    \item Gate-Source-Leckstrom: $I_\mathrm{GSS} = \SI{100}{\nano\ampere}$
    \item Leckstrom der Bootstrap-Diode: $I_\mathrm{LKDIODE} = \SI{10}{\micro\ampere}$
\end{itemize}
Der Leckstrom des Kondensators selbst ($I_\mathrm{LKCAP}$) wird vernachlässigt, da ein Keramikkondensator mit vernachlässigbarer Selbstentladung zum Einsatz kommt. Die Summe der Leckströme beträgt somit $I_\mathrm{Leak} \approx \SI{115,1}{\micro\ampere}$. Über die Einschaltdauer $t_\mathrm{on}$ resultiert daraus ein Ladungsverlust von:
\begin{equation}
Q_\mathrm{Leak} = I_\mathrm{Leak} \cdot t_\mathrm{on} = \SI{115,1}{\micro\ampere} \cdot \SI{50}{\micro\second} \approx \SI{5,8}{\nano\coulomb}
\end{equation}
Unter Berücksichtigung der für Hochvolt-Treiber typischen Ladung des Level-Shifters von $Q_\mathrm{LS} = \SI{3}{\nano\coulomb}$ ergibt sich die Gesamtladung zu:
\begin{equation}
Q_\mathrm{Total} = Q_\mathrm{Gate} + Q_\mathrm{LS} + Q_\mathrm{Leak} = \SI{111}{\nano\coulomb} + \SI{3}{\nano\coulomb} + \SI{5,8}{\nano\coulomb} = \SI{119,8}{\nano\coulomb}
\end{equation}

Um eine sichere Ansteuerung des High-Side-MOSFETs zu gewährleisten, wird ein maximaler Spannungsabfall (Ripple) von $\Delta V_\mathrm{BOOT} = \SI{1}{\volt}$ definiert. Eingesetzt in Gleichung \ref{eq:cboot_def} folgt daraus der minimal notwendige Kapazitätswert:
\begin{equation}
C_\mathrm{BOOT,min} = \frac{\SI{119,8}{\nano\coulomb}}{\SI{1}{\volt}} = \SI{119,8}{\nano\farad}
\end{equation}
In der praktischen Anwendung wird empfohlen, den Kondensator so zu dimensionieren, dass er nicht zu groß gewählt wird, um die Ladezeit in kurzen Low-Side-Phasen nicht unnötig zu verlängern, aber ausreichend Reserve bietet. Ein Normwert von \SI{150}{\nano\farad} oder \SI{220}{\nano\farad} ist für diese Anwendung angemessen.



%----------


\chapter{Hardware-Entwurf und Dimensionierung}
\label{chap:hardware}

Ziel des Hardware-Entwurfs ist die Realisierung einer robusten 3-Phasen-Brückenschaltung (B6-Topologie), die für einen Nennstrom von \SI{30}{\ampere} bei einer Zwischenkreisspannung von \SI{36}{\volt} ausgelegt ist. Im folgenden Kapitel werden die zentralen Komponenten die Leistungshalbleiter und deren Ansteuerung ausgewählt und dimensioniert.

\section{Hardwareaufbau des Prototpys}
\begin{figure}[htbp]
    \centering
    \begin{tikzpicture}
        % Paths, nodes and wires:
        \node[shape=rectangle, draw, line width=1pt, minimum width=3.965cm, minimum height=1.965cm] at (6, 5){};
        \draw (4.5, 6) -- (4.5, 7);
        \draw (5.5, 6) -- (5.5, 7);
        \draw (6.5, 6) -- (6.5, 7);
        \draw (7.5, 6) -- (7.5, 7);
        \draw (4.5, 3) -- (4.5, 4);
        \draw (5.5, 3) -- (5.5, 4);
        \draw (6.5, 3) -- (6.5, 4);
        \draw (7.5, 3) -- (7.5, 4);
        \node[shape=rectangle, minimum width=0.965cm, minimum height=0.965cm] at (4.5, 5.5){} node[anchor=center, align=center, text width=0.577cm, inner sep=6pt] at (4.5, 5.5){GND};
        \node[shape=rectangle, minimum width=0.965cm, minimum height=0.965cm] at (5.5, 5.5){} node[anchor=center, align=center, text width=0.577cm, inner sep=6pt] at (5.5, 5.5){LIN*};
        \node[shape=rectangle, minimum width=0.965cm, minimum height=0.965cm] at (6.5, 5.5){} node[anchor=center, align=center, text width=0.577cm, inner sep=6pt] at (6.5, 5.5){HIN};
        \node[shape=rectangle, minimum width=0.965cm, minimum height=0.965cm] at (7.5, 5.5){} node[anchor=center, align=center, text width=0.577cm, inner sep=6pt] at (7.5, 5.5){VCC};
        \node[shape=rectangle, minimum width=0.965cm, minimum height=0.965cm] at (4.5, 4.5){} node[anchor=center, align=center, text width=0.577cm, inner sep=6pt] at (4.5, 4.5){LO};
        \node[shape=rectangle, minimum width=0.965cm, minimum height=0.965cm] at (5.5, 4.5){} node[anchor=center, align=center, text width=0.577cm, inner sep=6pt] at (5.5, 4.5){VS};
        \node[shape=rectangle, minimum width=0.965cm, minimum height=0.965cm] at (6.5, 4.5){} node[anchor=center, align=center, text width=0.577cm, inner sep=6pt] at (6.5, 4.5){HO};
        \node[shape=rectangle, minimum width=0.965cm, minimum height=0.965cm] at (7.5, 4.5){} node[anchor=center, align=center, text width=0.577cm, inner sep=6pt] at (7.5, 4.5){VB};
        \node[shape=circle, draw, line width=1pt, minimum width=0.465cm] at (3.677, 6.323){};
        \draw (9, 6) to[empty diode] (9, 4);
        \draw (9, 6) -| (9, 7) -- (7.5, 7);
        \draw (7.5, 3) -- (9, 3) -| (9, 4);
        \node[sground, xscale=-1, yscale=-1] at (4.5, 7){};
        \draw (7.5, 2) to[capacitor, l={$220\space\mathrm{nF} /63\space\mathrm{V}$}] (5.5, 2);
        \draw (5.5, 2) -| (5.5, 3);
        \draw (7.5, 2) -| (7.5, 3);
    \end{tikzpicture}
    \caption{}
    \label{fig:prototype}
\end{figure}

\section{Dimensionierung der Leistungsendstufe}
Die Auswahl der Schaltelemente ist entscheidend für den Wirkungsgrad und die thermische Stabilität des Controllers. Aufgrund der Anforderungen an hohe Schaltgeschwindigkeiten und geringe Verluste kommen N-Kanal-MOSFETs zum Einsatz.

\subsection{Auswahl des MOSFETs}
Für diese Anwendung fiel die Wahl auf den \textbf{Infineon IPP034N08N5} im TO-220-Gehäuse. Diese Entscheidung basiert auf einer Analyse der kritischen Parameter Spannungsfestigkeit und Einschaltwiderstand.

Da die Ladeschlussspannung des \SI{36}{\volt}-Akkus bis zu \SI{42}{\volt} betragen kann und beim generatorischen Bremsen oder durch parasitäre Induktivitäten Spannungsspitzen auftreten, ist eine Sperrspannung ($V_{DS}$) deutlich oberhalb der Batteriespannung notwendig. Der IPP034N08N5 bietet mit \SI{80}{\volt} eine Sicherheitsreserve von fast Faktor 2, was den Verzicht auf komplexe Netzwerke ermöglicht.

Ein weiteres entscheidendes Kriterium ist der statische Drain-Source-Einschaltwiderstand ($R_{DS(on)}$). Mit einem Maximalwert von \SI{3,4}{\milli\ohm} gehört dieser Transistor zu den effizientesten seiner Klasse. Ein geringer $R_{DS(on)}$ ist essenziell, da die Durchlassverluste quadratisch mit dem Laststrom steigen.

Die wesentlichen Kennwerte sind in Tabelle \ref{tab:mosfet_data} zusammengefasst.

\begin{table}[h!]
    \centering
    \caption{Zentrale Parameter des IPP034N08N5}
    \label{tab:mosfet_data}
    \begin{tabular}{l l r}
        \toprule
        Parameter & Symbol & Wert \\
        \midrule
        Max. Drain-Source-Spannung & $V_{DS}$ & \SI{80}{\volt} \\
        Max. Dauerstrom ($T_C = 25^\circ C$) & $I_D$ & \SI{120}{\ampere} \\
        Einschaltwiderstand & $R_{DS(on)}$ & \SI{3,4}{\milli\ohm} \\
        Gate-Ladung (0..\SI{10}{\volt}) & $Q_G$ & \SI{69}{\nano\coulomb} \\
        Eingangskapazität & $C_{iss}$ & \SI{4800}{\pico\farad} \\
        \bottomrule
    \end{tabular}
\end{table}

\begin{figure}[htbp]
    \centering
    \begin{tikzpicture}
        % Paths, nodes and wires:
        \node[shape=circle, draw, line width=0.5pt, minimum width=1.429cm] at (5.414, 9.268){};
        \node[nigfete, bodydiode] at (5.5, 9.27){};
        \node[shape=rectangle, minimum width=0.926cm, minimum height=0.544cm](N1) at (3.196, 9.013){} node[anchor=center] at (N1.text){$\mathrm{Gate}$};
        \node[shape=rectangle, minimum width=0.926cm, minimum height=0.544cm](N2) at (6.243, 8.135){} node[anchor=center] at (N2.text){$\mathrm{Source}$};
        \node[shape=rectangle, minimum width=0.926cm, minimum height=0.544cm](N3) at (6.176, 10.359){} node[anchor=center] at (N3.center){$\mathrm{Drain}$};
        \node[ocirc] at (3.952, 8.997){};
        \node[ocirc] at (5.502, 7.943){};
        \node[ocirc] at (5.501, 10.554){};
        \draw (5.5, 10.04) -| (5.5, 10.5);
        \draw (5.5, 8.5) -- (5.5, 8);
        \draw (4.5, 9) -- (4, 9);
    \end{tikzpicture}
    \caption{Symbol des verwendeten N-Kanal MOSFETs mit interner Freilaufdiode.}
    \label{fig:mosfet_symbol}
\end{figure}

\subsection{Berechnung der Verlustleistung}
Um die thermische Auslegung in Kapitel \ref{chap:thermik} vorzubereiten, werden die zu erwartenden Verluste im Nennbetrieb abgeschätzt. Die Gesamtverluste setzen sich aus den Durchlassverlusten ($P_{Cond}$) und den Schaltverlusten ($P_{SW}$) zusammen.

Die statischen Durchlassverluste lassen sich über das Ohmsche Gesetz herleiten. Bei einem angenommenen maximalen Phasenstrom von $I_{rms} = \SI{30}{\ampere}$ ergibt sich pro Schalter:
\begin{equation}
    P_{Cond} = I_{rms}^2 \cdot R_{DS(on)} = (\SI{30}{\ampere})^2 \cdot \SI{0,0034}{\ohm} = \SI{3,06}{\watt}
\end{equation}

Dies stellt den dominanten Anteil der Verluste dar. Hinzu kommen die Schaltverluste, die durch das nicht-ideale Schalten (gleichzeitiges Anliegen von Strom und Spannung während der Umschaltphasen) entstehen. Diese werden maßgeblich durch die Gate-Treiber-Stufe und die gewählte Schaltfrequenz beeinflusst.

\section{Auslegung der Gate-Treiber-Stufe}
Da N-Kanal-MOSFETs in der High-Side-Position einer Brückenschaltung eine Ansteuerspannung oberhalb des Versorgungspotentials benötigen ($V_G > V_S + V_{th}$), ist der Einsatz spezialisierter Halbbrückentreiber erforderlich. Für dieses Design wurde der \textbf{IR2104} ausgewählt.

Der Baustein übernimmt zwei zentrale Aufgaben: Zum einen generiert er mittels einer internen Logik die notwendige Totzeit (Deadtime) von typisch \SI{520}{\nano\second}, um ein gleichzeitiges Leiten beider Transistoren (Shoot-Through) hardwareseitig zu unterbinden. Zum anderen ermöglicht er über eine externe Bootstrap-Beschaltung die Versorgung des High-Side-Schalters.

\subsection{Dimensionierung der Bootstrap-Kapazität}
Ein kritischer Aspekt des Designs ist die Dimensionierung des Bootstrap-Kondensators ($C_{BS}$). Dieser Kondensator fungiert als schwimmende Spannungsquelle für den High-Side-Treiber. Er muss genügend Ladung $Q_{Total}$ speichern, um das Gate des MOSFETs vollständig umzuladen und Leckströme während der Leitphase zu kompensieren, ohne dass die Spannung signifikant einbricht.

Die benötigte Gesamtladung $Q_{Total}$ setzt sich additiv aus drei Komponenten zusammen: der Gateladung $Q_{Gate}$, dem Verbrauch des Level-Shifters $Q_{LS}$ und den Verlusten durch Leckströme $Q_{Leak}$.

\subsubsection*{1. Ermittlung der Gateladung}
Der Treiber wird mit einer Spannung von $V_{CC} = \SI{15}{\volt}$ betrieben. Da das Datenblatt des MOSFETs die Gateladung nur für \SI{10}{\volt} spezifiziert, muss der Wert extrapoliert werden. Unter Berücksichtigung der Eingangskapazität $C_{iss}$ ergibt sich eine effektive Ladungsmenge von:
\begin{equation}
    Q_{Gate} \approx Q_{G(10V)} + (V_{CC} - \SI{10}{\volt}) \cdot C_{iss} \approx \SI{111}{\nano\coulomb}
\end{equation}

\subsubsection*{2. Leckströme und Gesamtladung}
Zusätzlich muss der Kondensator die Leckströme der Schaltung während der maximalen Einschaltdauer ($t_{on}$) puffern. Für den Worst-Case ($f_s = \SI{20}{\kilo\hertz} \rightarrow t_{on} = \SI{50}{\micro\second}$) summieren sich die Leckströme von Treiber, Diode und Kondensator auf ca. \SI{115}{\micro\ampere}. Dies entspricht einem Ladungsverlust von $Q_{Leak} \approx \SI{5,8}{\nano\coulomb}$.

Mit einer Reserve für den Level-Shifter ($Q_{LS} \approx \SI{3}{\nano\coulomb}$) resultiert eine Gesamtladung von:
\begin{equation}
    Q_{Total} = \SI{111}{\nano\coulomb} + \SI{5,8}{\nano\coulomb} + \SI{3}{\nano\coulomb} \approx \SI{120}{\nano\coulomb}
\end{equation}

\subsubsection*{3. Ergebnis}
Um sicherzustellen, dass die Gate-Spannung auch am Ende der Einschaltdauer stabil bleibt, wird ein maximal zulässiger Spannungsabfall (Ripple) von $\Delta V_{BS} = \SI{1}{\volt}$ definiert. Die Mindestkapazität berechnet sich somit zu:
\begin{equation}
    C_{BS,min} = \frac{Q_{Total}}{\Delta V_{BS}} = \frac{\SI{120}{\nano\coulomb}}{\SI{1}{\volt}} = \SI{120}{\nano\farad}
\end{equation}

Um Toleranzen und Alterungseffekte (DC-Bias bei Keramikkondensatoren) auszugleichen, wird der nächsthöhere Normwert gewählt. Es kommt ein \textbf{\SI{220}{\nano\farad}} Keramikkondensator (X7R) zum Einsatz.

\section{Zwischenkreis-Auslegung}
Der Zwischenkreis (DC-Link) dient als Energiespeicher und stabilisiert die Spannung bei den hochfrequenten Schaltvorgängen.
\todo[inline]{Hier folgt später der Text zu den Elkos.}