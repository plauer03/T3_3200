% ===================================================================
% KAPITEL 1: EINLEITUNG
% ===================================================================
\chapter{Einleitung}
\todo[inline, color=green!40]{Zielumfang: ca. 2 Seiten}

\section{Motivation}
\todo[inline]{Beschreiben: Trend zur E-Mobilität, Anforderungen an hochintegrierte Antriebe im E-Kart-Bereich.}

\section{Zielsetzung}
Ziel dieser Arbeit ist die Entwicklung und Validierung einer Motorsteuerung für einen 36V BLDC-Motor. Der Controller soll für einen Dauerphasenstrom von 30A ausgelegt werden und mittels Space Vector PWM (SVPWM) angesteuert werden.

\section{Abgrenzung}
\todo[inline]{Fokus: Hardware-Design (Leistungselektronik) und Open-Loop SVPWM. Keine feldorientierte Regelung (FOC) mit Stromrückführung in dieser Phase.}

% ===================================================================
% KAPITEL 2: GRUNDLAGEN
% ===================================================================
\chapter{Theoretische Grundlagen der Ansteuerung}
\todo[inline, color=green!40]{Zielumfang: ca. 4-5 Seiten. Hier mathematisches Verständnis beweisen!}

\section{Der bürstenlose Gleichstrommotor (BLDC)}
\todo[inline]{Kurz: Statoraufbau, Rotorlageerkennung via Hall-Sensoren. Ersatzschaltbild (R, L, BEMF).}

\section{Topologie des B6-Wechselrichters}
Die Ansteuerung erfolgt über eine 3-phasige Brückenschaltung (B6-Topologie).
\todo[inline]{Erklären: Kommutierungszelle. Was passiert in der Totzeit? (Freilaufdioden-Strom).}



\section{Raumzeigermodulation (SVPWM)}
Im Gegensatz zur blockförmigen Kommutierung ermöglicht die SVPWM eine bessere Ausnutzung der Zwischenkreisspannung um den Faktor $2/\sqrt{3} \approx 1,15$.
\todo[inline]{Das Hexagon-Diagramm einfügen. Sektoren erklären.}

Die Einschaltzeiten $T_1, T_2$ und $T_0$ (Nullzeiger) für einen Referenzvektor $V_{ref}$ im Sektor 1 berechnen sich trigonometrisch:
\begin{equation}
    T_1 = \frac{\sqrt{3} \cdot T_s \cdot |V_{ref}|}{V_{DC}} \cdot \sin\left(\frac{\pi}{3} - \theta\right)
\end{equation}
\begin{equation}
    T_2 = \frac{\sqrt{3} \cdot T_s \cdot |V_{ref}|}{V_{DC}} \cdot \sin(\theta)
\end{equation}
\begin{equation}
    T_0 = T_s - T_1 - T_2
\end{equation}

% ===================================================================
% KAPITEL 3: HARDWARE DESIGN (KERNSTÜCK)
% ===================================================================
\chapter{Hardware-Entwurf und Dimensionierung}
\todo[inline, color=red!40]{KERNSTÜCK (8-10 Seiten). Hier Datenblätter und Application Notes referenzieren!}

\section{Auswahl der Leistungshalbleiter}
Für die 36V-Anwendung (max. 42V Ladespannung) wurde der \textbf{Infineon IPP034N08N5} (80V, TO-220) gewählt.
\todo[inline]{Begründung: 80V bietet genügend Sicherheitsmarge für induktive Spikes. $R_{DS(on)}$ ist extrem niedrig.}

\subsection{Berechnung der Leitendverluste}
Basierend auf der Infineon Application Note AN 2015-05 lassen sich die statischen Verluste $P_{Cond}$ abschätzen. Bei einem Phasenstrom von $I_{rms} \approx 30A$ und $R_{DS(on)} = 3,4m\Omega$ gilt pro MOSFET:
\begin{equation}
    P_{Cond} = I_{rms}^2 \cdot R_{DS(on)} = (30A)^2 \cdot 0,0034\Omega = 3,06 W
\end{equation}
\todo[inline]{Diskutieren: 3W sind beherrschbar, erfordern aber Kühlung (siehe Kap 4).}

\subsection{Schaltverluste}
\todo[inline]{Formel für $P_{SW}$ (Switching Losses) aus AN einfügen. Abhängig von Schaltfrequenz $f_{sw}$ und Rise/Fall-Time.}

\section{Gate-Treiber und Bootstrap-Dimensionierung}
Als Treiber kommt der \textbf{IR2104} zum Einsatz. Die Versorgung der High-Side erfolgt über eine Bootstrap-Schaltung.

\subsection{Berechnung des Bootstrap-Kondensators}
Die benötigte Ladung $Q_{Total}$ muss vom Kondensator $C_{BS}$ bereitgestellt werden, ohne dass die Spannung zu stark einbricht.
\begin{equation}
    Q_{Total} = Q_{Gate} + Q_{LS} + (I_{Leak} \cdot t_{on})
\end{equation}
Mit $Q_{Gate} \approx 69nC$ (aus Datenblatt IPP034N08N5) und $Q_{LS} \approx 3nC$ sowie Sicherheitsfaktoren ergibt sich:
\begin{equation}
    C_{BS} \geq \frac{Q_{Total}}{\Delta V_{Ripple}}
\end{equation}
\todo[inline]{Berechnung einfügen: Gewählt 220nF - 470nF (X7R Keramik).}
\todo[inline]{Erwähnen: Bootstrap-Diode UF4007 (Ultra Fast) statt 1N4007, da $t_{rr}$ kritisch ist bei hohen Frequenzen.}



\section{Zwischenkreis (DC-Link)}
\todo[inline]{Auslegung der Elkos für den Ripplestrom. Parallelschaltung von Keramik-Kondensatoren zur Filterung hoher Frequenzen.}

% ===================================================================
% KAPITEL 4: THERMIK
% ===================================================================
\chapter{Thermisches Management}
\todo[inline, color=green!40]{Ingenieurs-Aspekt: Kühlkörperberechnung.}

\section{Thermisches Ersatzschaltbild}
Um die Junction-Temperatur $T_J$ unter dem Maximum ($175^\circ C$) zu halten, gilt:
\begin{equation}
    T_J = P_{tot} \cdot (R_{thJC} + R_{thCH} + R_{thHA}) + T_{Amb}
\end{equation}
Dabei ist:
\begin{itemize}
    \item $R_{thJC}$: Wärmewiderstand Chip-Gehäuse (aus Datenblatt: $1,1 K/W$).
    \item $R_{thCH}$: Wärmewiderstand Gehäuse-Kühlkörper (Isolierscheibe! Glimmer $\approx 0,5 K/W$).
    \item $R_{thHA}$: Wärmewiderstand Kühlkörper-Luft (zu berechnen).
\end{itemize}

\section{Auslegung des Kühlkörpers}
\todo[inline]{Rechnung: Welchen Rth-Wert muss der Kühlkörper haben, um bei 30A (ca. 4-5W Gesamtverlust pro FET) stabil zu bleiben?}

% ===================================================================
% KAPITEL 5: PCB DESIGN
% ===================================================================
\chapter{PCB-Design und Layout}
\section{Masseführung}
\todo[inline]{Trennung von Power-GND und Signal-GND (Sternpunkt beim Elko).}
\section{Strompfade}
\todo[inline]{Minimierung der Leiterschleifen in der Kommutierungszelle zur Reduktion parasitärer Induktivitäten.}

% ===================================================================
% KAPITEL 6: IMPLEMENTIERUNG
% ===================================================================
\chapter{Software-Implementierung und Validierung}
\section{STM32 Konfiguration}
\todo[inline]{Timer-Setup: Center-Aligned PWM für SVPWM. ADC-Triggerung in der PWM-Mitte.}
\section{Messergebnisse}
\todo[inline]{Platzhalter für Oszi-Bilder: Gate-Signale, Totzeit-Überprüfung, Phasenstrom.}

% ===================================================================
% KAPITEL 7: FAZIT
% ===================================================================
\chapter{Fazit und Ausblick}
\todo[inline]{Zusammenfassung. Nächste Schritte: Strommessung integrieren, FOC-Regelung.}