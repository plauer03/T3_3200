% settings.tex
% -------------------------------------------------------------------
% PAKETE & EINSTELLUNGEN
% -------------------------------------------------------------------
\usepackage[utf8]{inputenc}
\usepackage[T1]{fontenc}
\usepackage{lmodern}
\usepackage[ngerman]{babel}

% Typografie & Mathe
\usepackage{amsmath, amssymb, amsthm}
\usepackage[nopatch=footnote]{microtype}
\usepackage[onehalfspacing]{setspace} % 1.5 Zeilenabstand

% Layout & Seitenränder (DHBW konform)
\usepackage[
  inner=3cm,      % Bundsteg
  outer=2.5cm, 
  top=2.5cm, 
  bottom=2.5cm
]{geometry}

% Kopf- und Fußzeilen
\usepackage[automark, headsepline]{scrlayer-scrpage}
\clearpairofpagestyles
\ihead{\headmark}
\ohead{\pagemark}
\pagestyle{scrheadings}

% Einheiten (WICHTIG für Ihre Berechnungen!)
\usepackage{siunitx}
\sisetup{
  locale = DE, 
  per-mode = symbol,
  detect-all
}

%Cirkuitz
\usepackage[european, straightvoltages]{circuitikz}
\usepackage{tikz} 

% Grafiken & Tabellen
\usepackage{graphicx}
\graphicspath{{images/}}
\usepackage{booktabs} % Schöne Tabellenlinien

% Literatur
\usepackage{csquotes}
\usepackage[backend=biber, style=ieee]{biblatex}
\addbibresource{references.bib}

% Verlinkungen (als letztes laden)
\usepackage[hidelinks]{hyperref}
\usepackage[colorinlistoftodos, prependcaption, textsize=small]{todonotes}

% Metadaten-Kommandos
\newcommand{\titel}{Entwicklung eines BLDC-Motorcontrollers}
\newcommand{\projektart}{Studienarbeit II (T3\_3200)}
\newcommand{\studiengang}{Elektrotechnik}
\newcommand{\hochschule}{Duale Hochschule Baden-Württemberg Karlsruhe}
\newcommand{\firma}{ARKU Maschinenbau GmbH}
\newcommand{\autor}{Pascal Lauer}
\newcommand{\matrikel}{5558179}       
\newcommand{\kurs}{TEL23AT}
\newcommand{\abgabedatum}{09.04.2026}
\newcommand{\bearbeitungszeitraum}{12 Wochen}
\newcommand{\firmaadresse}{76532 Baden-Baden}
%\newcommand{\betreuerfirma}{B. Eng. Adrian Fleig}
\newcommand{\betreuerfirma}{Prof. Dr. Markus Bell}
\newcommand{\gutachterdh}{Titel Vorname Nachname}