% -------------------------------------------------------------------
% settings.tex - Layout, Pakete & Typografie
% -------------------------------------------------------------------

% --- KOPF- UND FUSSZEILEN ---
\usepackage[automark]{scrlayer-scrpage}

% Löscht alle Standardvorgaben
\clearpairofpagestyles 

% 1. KOPFZEILE: Rechts den aktuellen Abschnitt (Section)
% Das [\headmark]{\headmark} sorgt dafür, dass es (falls gewünscht) 
% auch auf Kapitelseiten erscheint. Normalerweise lässt man das [] leer.
\ihead{\textbf{T3\_3200} | Pascal Lauer}
\ohead{\headmark} 

% 2. FUSSZEILE: Rechts die Seitenzahl
% WICHTIG: Das [\pagemark]{\pagemark} sorgt dafür, dass die Seitenzahl 
% AUCH auf Inhaltsverzeichnis- und Kapitelstartseiten (Stil "plain") steht!
\ofoot[\pagemark]{\pagemark} 

% 3. LINIE: Trennlinie unter dem Header
\KOMAoptions{headsepline=true}

% 4. SCHRIFTART: Header/Footer an das Dokument anpassen (Helvet/Serifenlos)
\setkomafont{pageheadfoot}{\normalfont\sffamily\fontsize{12pt}{14pt}\selectfont}
\setkomafont{pagenumber}{\normalfont\sffamily\fontsize{12pt}{14pt}\selectfont}


% --- GRUNDLAGEN & SPRACHE ---
\usepackage{layout} 
\usepackage{geometry}
\geometry{
   left=2.5cm,
   right=2.5cm,
	 top=3cm,
	 bottom=3cm}
\usepackage{setspace}
\onehalfspacing % 1,5 Zeilenabstand 		
\setlength{\parindent}{0pt} % nicht einrücken

\usepackage{caption} % fett gedruckt Abbildung/Tabelle
\captionsetup[figure]{labelfont=bf,textfont=normalfont}
\captionsetup[table]{labelfont=bf,textfont=normalfont}


\usepackage[utf8]{inputenc}
\usepackage[T1]{fontenc}
\usepackage[ngerman]{babel}
\usepackage[nopatch=footnote]{microtype} % Typografische Feinheiten
\setlength{\marginparwidth}{2cm}


\usepackage{epsfig}		% needed to import eps files
\usepackage{lipsum}		% needed to generate blindtext
\usepackage{libertine}
\usepackage{libertinust1math}
\usepackage{amssymb}
\usepackage{amsmath}
\renewcommand{\familydefault}{\sfdefault}
% \usepackage{sfmath} % full on arial


% --- MATHE & TECHNIK ---
\usepackage{amsmath, amssymb}
\usepackage{siunitx}
\sisetup{locale = DE, per-mode = symbol, detect-all}
\usepackage{tikz}
\usepackage[european, straightvoltages]{circuitikz}

% --- GRAFIKEN & TABELLEN ---
\usepackage{graphicx}
\graphicspath{{images/}}
\usepackage{booktabs}
\usepackage{pdfpages} 
\usepackage{booktabs}

% --- LITERATUR (IEEE) ---
\usepackage{csquotes}
\usepackage[
  backend=biber, 
  style=ieee, 
  dashed=false
]{biblatex}
\addbibresource{references.bib}

% --- SONSTIGES ---
\usepackage[colorlinks=true, allcolors=black, hidelinks]{hyperref}
\usepackage[colorinlistoftodos, textsize=tiny]{todonotes}


% Metadaten-Kommandos
\newcommand{\titel}{Entwicklung eines BLDC-Motorcontrollers}
\newcommand{\projektart}{Studienarbeit II (T3\_3200)}
\newcommand{\studiengang}{Elektrotechnik}
\newcommand{\hochschule}{Duale Hochschule Baden-Württemberg Karlsruhe}
\newcommand{\firma}{ARKU Maschinenbau GmbH}
\newcommand{\autor}{Pascal Lauer}
\newcommand{\matrikel}{5558179}       
\newcommand{\kurs}{TEL23AT}
\newcommand{\abgabedatum}{09.04.2026}
\newcommand{\bearbeitungszeitraum}{12 Wochen}
\newcommand{\firmaadresse}{76532 Baden-Baden}
%\newcommand{\betreuerfirma}{B. Eng. Adrian Fleig}
\newcommand{\betreuerfirma}{Prof. Dr. Markus Bell}
\newcommand{\gutachterdh}{Titel Vorname Nachname}

